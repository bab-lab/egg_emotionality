\PassOptionsToPackage{unicode=true}{hyperref} % options for packages loaded elsewhere
\PassOptionsToPackage{hyphens}{url}
%
\documentclass[]{book}
\usepackage{lmodern}
\usepackage{amssymb,amsmath}
\usepackage{ifxetex,ifluatex}
\usepackage{fixltx2e} % provides \textsubscript
\ifnum 0\ifxetex 1\fi\ifluatex 1\fi=0 % if pdftex
  \usepackage[T1]{fontenc}
  \usepackage[utf8]{inputenc}
  \usepackage{textcomp} % provides euro and other symbols
\else % if luatex or xelatex
  \usepackage{unicode-math}
  \defaultfontfeatures{Ligatures=TeX,Scale=MatchLowercase}
\fi
% use upquote if available, for straight quotes in verbatim environments
\IfFileExists{upquote.sty}{\usepackage{upquote}}{}
% use microtype if available
\IfFileExists{microtype.sty}{%
\usepackage[]{microtype}
\UseMicrotypeSet[protrusion]{basicmath} % disable protrusion for tt fonts
}{}
\IfFileExists{parskip.sty}{%
\usepackage{parskip}
}{% else
\setlength{\parindent}{0pt}
\setlength{\parskip}{6pt plus 2pt minus 1pt}
}
\usepackage{hyperref}
\hypersetup{
            pdftitle={Inside Out - Physiology of Emotional Reactivity},
            pdfborder={0 0 0},
            breaklinks=true}
\urlstyle{same}  % don't use monospace font for urls
\usepackage{longtable,booktabs}
% Fix footnotes in tables (requires footnote package)
\IfFileExists{footnote.sty}{\usepackage{footnote}\makesavenoteenv{longtable}}{}
\usepackage{graphicx,grffile}
\makeatletter
\def\maxwidth{\ifdim\Gin@nat@width>\linewidth\linewidth\else\Gin@nat@width\fi}
\def\maxheight{\ifdim\Gin@nat@height>\textheight\textheight\else\Gin@nat@height\fi}
\makeatother
% Scale images if necessary, so that they will not overflow the page
% margins by default, and it is still possible to overwrite the defaults
% using explicit options in \includegraphics[width, height, ...]{}
\setkeys{Gin}{width=\maxwidth,height=\maxheight,keepaspectratio}
\setlength{\emergencystretch}{3em}  % prevent overfull lines
\providecommand{\tightlist}{%
  \setlength{\itemsep}{0pt}\setlength{\parskip}{0pt}}
\setcounter{secnumdepth}{5}
% Redefines (sub)paragraphs to behave more like sections
\ifx\paragraph\undefined\else
\let\oldparagraph\paragraph
\renewcommand{\paragraph}[1]{\oldparagraph{#1}\mbox{}}
\fi
\ifx\subparagraph\undefined\else
\let\oldsubparagraph\subparagraph
\renewcommand{\subparagraph}[1]{\oldsubparagraph{#1}\mbox{}}
\fi

% set default figure placement to htbp
\makeatletter
\def\fps@figure{htbp}
\makeatother

\usepackage{booktabs}
\usepackage{amsthm}
\makeatletter
\def\thm@space@setup{%
  \thm@preskip=8pt plus 2pt minus 4pt
  \thm@postskip=\thm@preskip
}
\makeatother
\usepackage[]{natbib}
\bibliographystyle{apalike}

\title{Inside Out - Physiology of Emotional Reactivity}
\author{}
\date{\vspace{-2.5em}}

\begin{document}
\maketitle

{
\setcounter{tocdepth}{1}
\tableofcontents
}
\hypertarget{section}{%
\chapter{}\label{section}}

\begin{figure}
\centering
\includegraphics{images/inside_out_banner.png}
\caption{}
\end{figure}

Introduction

It is no surprise that often people experience strong emotional responses in their bodies. However, physical symptoms, particularly gastrointestinal symptoms, have scarcely been studied as a measure of emotional arousal. This study will explore the relationship between physical and emotional symptoms and health. In addition, we will explore the impact of the COVID-19 pandemic on physical and emotional health.

\hypertarget{summary}{%
\section{Summary}\label{summary}}

The purpose of this two-part study is to explore the relationship between somatic symptoms and emotions in adults.

Recruitment target: N = 150.

Study 1 - COVID-19

Part 1 will be an online study that will explore the relationship between emotions and somatic symptamology in the wake of the COVID-19 pandemic, while exploring individual differences in social, environmental, personality, and lifestyle factors which may mitigate or exacerbate the negative psychological impact of this stressor.

Participants will complete a range of questionnaires assessing social and emotional functioning, physical health symptoms, early life adversity, physical health, and a range of questionnaires assessing the impact of COVID-19.

Study 2 - EGG

Part 2 will be an in-person study that will explore the relationship between gastrointestinal activity and emotions utilizing electrogastrography.

Participants who had previously completed Part 1 will return for an in-person session in the lab in which they will watch a series of sad, scary, and neutral movies while electrophysiology recordings are made. Then they will complete a range of questionnaires assessing social and emotional functioning, physical health symptoms, early life adversity, and physical health assessments.

\begin{center}\rule{0.5\linewidth}{0.5pt}\end{center}

\hypertarget{keywords}{%
\section{Keywords}\label{keywords}}

stress, emotions, mental health, somatic symptoms, egg, physiology

\begin{center}\rule{0.5\linewidth}{0.5pt}\end{center}

\hypertarget{background}{%
\section{Background}\label{background}}

While we often describe our emotions as ``gut feelings'', surprisingly little research has examined how emotions and the gastrointestinal system interact. Given the onset of a global pandemic, the situation provides a unique opportunity to investigate how an emotion inducing real-world event, COVID-19, might influence somatic symptoms and the stress response.

Prior research during public health crises, such as the SARS epidemic in 2006, reveal that the stress associated with quarantine during the epidemic was associated with higher symptoms of acute stress disorder and later post traumatic stress symptoms (Bai et al., 2004). However, research also reveals that in the wake of the SARS epidemic, individuals found their friends and family members more supportive (Lau, Yang, Tsui, Pang, \& Wing, 2006). Similarly, research suggests that social support obtained through social interactions after the events of September 11th, 2001 reduced college students' symptoms of both depression and physical illness (MacGeorge, Samter, Feng, Gillihan, \& Graves, 2004).

Given that gastrointestinal and mental heath problems are highly comorbid, with anxiety five times higher in individuals with irritable bowel syndrome (IBS) than in those with no IBS symptoms (Lee et al., 2009), gastrointestinal and somatic symptoms may serve as a useful indicator of emotional functioning, particularly during this period of heightened awareness of physical health amid the COVID-19 pandemic. For example, stress in early life can affect both emotional and gastrointestinal symptoms and functioning. One study demonstrated that previous adverse care experiences were associated with both increased anxiety and incidence of gastrointestinal symptoms in youth (Callaghan et al., 2019). In addition, early adversity was associated with changes in gastrointestinal microbiome diversity that were correlated with neural activation to emotional faces (Callaghan et al., 2019).

Physiological methods such as heart rate and sweat response are common indicators of emotional arousal, but the electrograstrogram (EGG) is seldom used in psychological research. In one study, researchers found that movie clips capturing emotions of fear, disgust, and sadness, elicited a greater EGG response relative to a neutral condition (Vianna \& Tranel, 2006).

Few studies have explored the way that individual differences including early stress, social support, media consumption, and lifestyle factors may mitigate or exacerbate the negative somatic psychological impact of a stressor as well as the way that these variables and emotional functioning may influence the EGG response to negatively emotionally arousing movie clips.

Using questionnaires and electrogastrography, we seek to investigate the relationship between somatic symptoms (particularly gastrointestinal symptoms), gastric myoelectrical activity, and emotional functioning, in the context of a public health crisis as well as during emotionally arousing movie clips. We also hope to explore factors that might influence gastrointestinal responses to emotional arousal, and whether and how physical sensations are associated with emotions and physiological responses. We also hope to evaluate how EGG sits with other physiological measures, such as heart rate and sweat response, in order to explore whether emotional patterning of physiological responses contribute to meaningful differences in emotion regulation, the stress response, and mental health.

\begin{center}\rule{0.5\linewidth}{0.5pt}\end{center}

\hypertarget{specific-aims}{%
\section{Specific Aims}\label{specific-aims}}

Study 1 - COVID-19

\begin{enumerate}
\def\labelenumi{\arabic{enumi}.}
\tightlist
\item
  Establish a relationship between psychological stress and somatic symptoms as assessed by our newly developed Somatic Symptoms of Negative Affect (SOMNA) Questionnaire.
\item
  Investigate the individual differences that might influence the somatic and psychological response to stress (such as early life stress, social support, media consumption, diet and exercise, lifestyle habits, trait variables, etc.)
\item
  Examine how specific somatic symptoms in the context of stress may relate to mental health
\end{enumerate}

Study 2 - EGG

\begin{enumerate}
\def\labelenumi{\arabic{enumi}.}
\tightlist
\item
  Establish a relationship between emotionally arousing stimuli and the EGG response
\item
  Investigate the individual differences that might influence the EGG response to stress (such as early life stress, current stress, trait variables, etc.)
\item
  Examine how the EGG response sits with other physiological indices (such as heart rate and sweat response)
\item
  Explore the ways in which physical sensations are associated with emotions and physiological responses
\end{enumerate}

\hypertarget{study-1---covid---methods}{%
\chapter{Study 1 - COVID - Methods}\label{study-1---covid---methods}}

\hypertarget{measures}{%
\section{Measures}\label{measures}}

\hypertarget{information}{%
\subsection{Information}\label{information}}

\begin{longtable}[]{@{}llll@{}}
\toprule
\begin{minipage}[b]{0.22\columnwidth}\raggedright
Title\strut
\end{minipage} & \begin{minipage}[b]{0.27\columnwidth}\raggedright
Name\strut
\end{minipage} & \begin{minipage}[b]{0.22\columnwidth}\raggedright
Description\strut
\end{minipage} & \begin{minipage}[b]{0.18\columnwidth}\raggedright
Reference\strut
\end{minipage}\tabularnewline
\midrule
\endhead
\begin{minipage}[t]{0.22\columnwidth}\raggedright
info\strut
\end{minipage} & \begin{minipage}[t]{0.27\columnwidth}\raggedright
Information questionnaire\strut
\end{minipage} & \begin{minipage}[t]{0.22\columnwidth}\raggedright
Assesses demographics, health, and location information\strut
\end{minipage} & \begin{minipage}[t]{0.18\columnwidth}\raggedright
Made by BABLab\strut
\end{minipage}\tabularnewline
\begin{minipage}[t]{0.22\columnwidth}\raggedright
demographics\strut
\end{minipage} & \begin{minipage}[t]{0.27\columnwidth}\raggedright
Demographics questionnaire\strut
\end{minipage} & \begin{minipage}[t]{0.22\columnwidth}\raggedright
Assesses socioeconomic status, employment, and commitments (e.g., volunteer work, child care)\strut
\end{minipage} & \begin{minipage}[t]{0.18\columnwidth}\raggedright
Made by BABLab\strut
\end{minipage}\tabularnewline
\begin{minipage}[t]{0.22\columnwidth}\raggedright
covid\_objective\strut
\end{minipage} & \begin{minipage}[t]{0.27\columnwidth}\raggedright
Objective impact of COVID-19\strut
\end{minipage} & \begin{minipage}[t]{0.22\columnwidth}\raggedright
Assesses the objective impact of COVID-19 including infection, quarantine, household, social distancing etc.\strut
\end{minipage} & \begin{minipage}[t]{0.18\columnwidth}\raggedright
Made by BABLab\strut
\end{minipage}\tabularnewline
\bottomrule
\end{longtable}

\hypertarget{affect}{%
\subsection{Affect}\label{affect}}

\begin{longtable}[]{@{}llll@{}}
\toprule
\begin{minipage}[b]{0.22\columnwidth}\raggedright
Title\strut
\end{minipage} & \begin{minipage}[b]{0.27\columnwidth}\raggedright
Name\strut
\end{minipage} & \begin{minipage}[b]{0.22\columnwidth}\raggedright
Description\strut
\end{minipage} & \begin{minipage}[b]{0.18\columnwidth}\raggedright
Reference\strut
\end{minipage}\tabularnewline
\midrule
\endhead
\begin{minipage}[t]{0.22\columnwidth}\raggedright
panas\strut
\end{minipage} & \begin{minipage}[t]{0.27\columnwidth}\raggedright
Positive and negative affect schedule\strut
\end{minipage} & \begin{minipage}[t]{0.22\columnwidth}\raggedright
Assesses current and retrospective (past week) degree of self-reported positive and negative affect through emotion words\strut
\end{minipage} & \begin{minipage}[t]{0.18\columnwidth}\raggedright
Watson, Clark, \& Tellegen, 1988\strut
\end{minipage}\tabularnewline
\begin{minipage}[t]{0.22\columnwidth}\raggedright
paq\strut
\end{minipage} & \begin{minipage}[t]{0.27\columnwidth}\raggedright
Perth alexithymia questionnaire\strut
\end{minipage} & \begin{minipage}[t]{0.22\columnwidth}\raggedright
Assesses all components of alexithymia (i.e., difficulty identifying and describing one's own feelings and having an externally orientated thinking style) across negative and positive emotions\strut
\end{minipage} & \begin{minipage}[t]{0.18\columnwidth}\raggedright
Preece et al., 2018\strut
\end{minipage}\tabularnewline
\bottomrule
\end{longtable}

\hypertarget{somatic}{%
\subsection{Somatic}\label{somatic}}

\begin{longtable}[]{@{}llll@{}}
\toprule
\begin{minipage}[b]{0.22\columnwidth}\raggedright
Title\strut
\end{minipage} & \begin{minipage}[b]{0.27\columnwidth}\raggedright
Name\strut
\end{minipage} & \begin{minipage}[b]{0.22\columnwidth}\raggedright
Description\strut
\end{minipage} & \begin{minipage}[b]{0.18\columnwidth}\raggedright
Reference\strut
\end{minipage}\tabularnewline
\midrule
\endhead
\begin{minipage}[t]{0.22\columnwidth}\raggedright
somna\strut
\end{minipage} & \begin{minipage}[t]{0.27\columnwidth}\raggedright
Somatic markers of negative affect\strut
\end{minipage} & \begin{minipage}[t]{0.22\columnwidth}\raggedright
Assesses physical sensations of anxiety and sadness, where they are located, and their intensity\strut
\end{minipage} & \begin{minipage}[t]{0.18\columnwidth}\raggedright
Made by BABLab\strut
\end{minipage}\tabularnewline
\begin{minipage}[t]{0.22\columnwidth}\raggedright
maia\strut
\end{minipage} & \begin{minipage}[t]{0.27\columnwidth}\raggedright
Multidimensional Assessment of Interoceptive Awareness\strut
\end{minipage} & \begin{minipage}[t]{0.22\columnwidth}\raggedright
Multidimensional self-report measure of interoceptive body awareness\strut
\end{minipage} & \begin{minipage}[t]{0.18\columnwidth}\raggedright
Mehling et al., 2012\strut
\end{minipage}\tabularnewline
\begin{minipage}[t]{0.22\columnwidth}\raggedright
hai\strut
\end{minipage} & \begin{minipage}[t]{0.27\columnwidth}\raggedright
Health anxiety inventory\strut
\end{minipage} & \begin{minipage}[t]{0.22\columnwidth}\raggedright
Assesses people's anxiety about health symptoms (hypochondriasis)\strut
\end{minipage} & \begin{minipage}[t]{0.18\columnwidth}\raggedright
Salkovskis et al., 2006\strut
\end{minipage}\tabularnewline
\begin{minipage}[t]{0.22\columnwidth}\raggedright
ss\strut
\end{minipage} & \begin{minipage}[t]{0.27\columnwidth}\raggedright
Somatic symptoms\strut
\end{minipage} & \begin{minipage}[t]{0.22\columnwidth}\raggedright
Assesses a range of somatic symptoms in adult participants\strut
\end{minipage} & \begin{minipage}[t]{0.18\columnwidth}\raggedright
Körber et al., 2011\strut
\end{minipage}\tabularnewline
\begin{minipage}[t]{0.22\columnwidth}\raggedright
pill\strut
\end{minipage} & \begin{minipage}[t]{0.27\columnwidth}\raggedright
Pennebaker inventory of limbid languidness\strut
\end{minipage} & \begin{minipage}[t]{0.22\columnwidth}\raggedright
Measures people's tendency to notice and report a braod array of physical symptoms and sensations\strut
\end{minipage} & \begin{minipage}[t]{0.18\columnwidth}\raggedright
Pennebaker, 1982\strut
\end{minipage}\tabularnewline
\begin{minipage}[t]{0.22\columnwidth}\raggedright
pedsql\_gi\strut
\end{minipage} & \begin{minipage}[t]{0.27\columnwidth}\raggedright
Pediatric Quality of life -- Gastrointestinal Symptoms Module\strut
\end{minipage} & \begin{minipage}[t]{0.22\columnwidth}\raggedright
Assess incidence of Gastrointestinal Symptoms and fatigue in children\strut
\end{minipage} & \begin{minipage}[t]{0.18\columnwidth}\raggedright
Varni et al., 2015\strut
\end{minipage}\tabularnewline
\begin{minipage}[t]{0.22\columnwidth}\raggedright
med\_check\strut
\end{minipage} & \begin{minipage}[t]{0.27\columnwidth}\raggedright
Medication checklist\strut
\end{minipage} & \begin{minipage}[t]{0.22\columnwidth}\raggedright
List of medications that participants are on -- needed for physiology analyses as well as verification of physical health issues\strut
\end{minipage} & \begin{minipage}[t]{0.18\columnwidth}\raggedright
Made by BABLab\strut
\end{minipage}\tabularnewline
\begin{minipage}[t]{0.22\columnwidth}\raggedright
gastrointestinal\_disorders\strut
\end{minipage} & \begin{minipage}[t]{0.27\columnwidth}\raggedright
Gastrointetsinal disorders questionnaire\strut
\end{minipage} & \begin{minipage}[t]{0.22\columnwidth}\raggedright
Assesses gastrointestinal issues, their frequency and intensity\strut
\end{minipage} & \begin{minipage}[t]{0.18\columnwidth}\raggedright
Made by BABLab\strut
\end{minipage}\tabularnewline
\begin{minipage}[t]{0.22\columnwidth}\raggedright
rome\strut
\end{minipage} & \begin{minipage}[t]{0.27\columnwidth}\raggedright
Rome IV criteria questionnaire\strut
\end{minipage} & \begin{minipage}[t]{0.22\columnwidth}\raggedright
Assesses the presence of symptoms which meet criteria for irritable bowel syndrome as stated by the Rome IV\strut
\end{minipage} & \begin{minipage}[t]{0.18\columnwidth}\raggedright
Made by BABLab\strut
\end{minipage}\tabularnewline
\begin{minipage}[t]{0.22\columnwidth}\raggedright
menstrual\_cycle\strut
\end{minipage} & \begin{minipage}[t]{0.27\columnwidth}\raggedright
Menstrual cycle questionnaire\strut
\end{minipage} & \begin{minipage}[t]{0.22\columnwidth}\raggedright
Assesses menstrual phase, which affects gastrointestinal responding, as well as medications which may affect menstrual phase such as oral contraceptive use\strut
\end{minipage} & \begin{minipage}[t]{0.18\columnwidth}\raggedright
Made by BABLab\strut
\end{minipage}\tabularnewline
\begin{minipage}[t]{0.22\columnwidth}\raggedright
psst\strut
\end{minipage} & \begin{minipage}[t]{0.27\columnwidth}\raggedright
Premenstrual symptoms screening tool\strut
\end{minipage} & \begin{minipage}[t]{0.22\columnwidth}\raggedright
Assesses premenstrual syndromes and criteria for premenstrual dysphoric disorder (pmdd) as well as premenstrual syndrome (pms)\strut
\end{minipage} & \begin{minipage}[t]{0.18\columnwidth}\raggedright
Steiner, Macdougall, \& Brown, 2003\strut
\end{minipage}\tabularnewline
\begin{minipage}[t]{0.22\columnwidth}\raggedright
bss\strut
\end{minipage} & \begin{minipage}[t]{0.27\columnwidth}\raggedright
Bristol stool scale\strut
\end{minipage} & \begin{minipage}[t]{0.22\columnwidth}\raggedright
Diagnostic medical stool designed to classify the form of human faeces into seven categories\strut
\end{minipage} & \begin{minipage}[t]{0.18\columnwidth}\raggedright
Bristol Royal Infirmary\strut
\end{minipage}\tabularnewline
\bottomrule
\end{longtable}

\hypertarget{stress}{%
\subsection{Stress}\label{stress}}

\begin{longtable}[]{@{}llll@{}}
\toprule
\begin{minipage}[b]{0.22\columnwidth}\raggedright
Title\strut
\end{minipage} & \begin{minipage}[b]{0.27\columnwidth}\raggedright
Name\strut
\end{minipage} & \begin{minipage}[b]{0.22\columnwidth}\raggedright
Description\strut
\end{minipage} & \begin{minipage}[b]{0.18\columnwidth}\raggedright
Reference\strut
\end{minipage}\tabularnewline
\midrule
\endhead
\begin{minipage}[t]{0.22\columnwidth}\raggedright
covid\_subjective\strut
\end{minipage} & \begin{minipage}[t]{0.27\columnwidth}\raggedright
Subjective impact of COVID-19\strut
\end{minipage} & \begin{minipage}[t]{0.22\columnwidth}\raggedright
Assesses the subjective impact of COVID-19 on well-being\strut
\end{minipage} & \begin{minipage}[t]{0.18\columnwidth}\raggedright
Made by BABLab\strut
\end{minipage}\tabularnewline
\begin{minipage}[t]{0.22\columnwidth}\raggedright
pss\strut
\end{minipage} & \begin{minipage}[t]{0.27\columnwidth}\raggedright
Perceived stress scale\strut
\end{minipage} & \begin{minipage}[t]{0.22\columnwidth}\raggedright
Examines how different situations affected feelings and perceived stress in the last month\strut
\end{minipage} & \begin{minipage}[t]{0.18\columnwidth}\raggedright
Cohen, Kamarck, \& Mermelstein, 1983\strut
\end{minipage}\tabularnewline
\begin{minipage}[t]{0.22\columnwidth}\raggedright
sasrq\strut
\end{minipage} & \begin{minipage}[t]{0.27\columnwidth}\raggedright
Stanford acute stress reaction questionnaire\strut
\end{minipage} & \begin{minipage}[t]{0.22\columnwidth}\raggedright
Assesses the psychological symptoms experienced in the aftermath of a traumatic event\strut
\end{minipage} & \begin{minipage}[t]{0.18\columnwidth}\raggedright
Cardeña et al., 2000\strut
\end{minipage}\tabularnewline
\begin{minipage}[t]{0.22\columnwidth}\raggedright
cte\strut
\end{minipage} & \begin{minipage}[t]{0.27\columnwidth}\raggedright
Childhood traumatic events questionnaire\strut
\end{minipage} & \begin{minipage}[t]{0.22\columnwidth}\raggedright
The Childhood Traumautic Events Questionnaire is a brief survey of six early traumatic experiences (death, divorce, violence, sexual abuse, illnesss, and upheaval)\strut
\end{minipage} & \begin{minipage}[t]{0.18\columnwidth}\raggedright
Pennebaker \& Susman, 2013\strut
\end{minipage}\tabularnewline
\begin{minipage}[t]{0.22\columnwidth}\raggedright
ctq\strut
\end{minipage} & \begin{minipage}[t]{0.27\columnwidth}\raggedright
Childhood trauma questionnaire\strut
\end{minipage} & \begin{minipage}[t]{0.22\columnwidth}\raggedright
Assesses the severity of emotional abuse and neglect, physical abuse and neglect and sexual abuse\strut
\end{minipage} & \begin{minipage}[t]{0.18\columnwidth}\raggedright
Bernstein, 1994\strut
\end{minipage}\tabularnewline
\begin{minipage}[t]{0.22\columnwidth}\raggedright
ccfq\strut
\end{minipage} & \begin{minipage}[t]{0.27\columnwidth}\raggedright
Cognitive control and flexibility questionnaire\strut
\end{minipage} & \begin{minipage}[t]{0.22\columnwidth}\raggedright
Measures an individual's percevied ability to exert control over intrusive, unwanted (negative) thoughts and emotions, and their ability to flexibly cope with a stressful situation\strut
\end{minipage} & \begin{minipage}[t]{0.18\columnwidth}\raggedright
Gabrys et al., 2018\strut
\end{minipage}\tabularnewline
\begin{minipage}[t]{0.22\columnwidth}\raggedright
ptgi\strut
\end{minipage} & \begin{minipage}[t]{0.27\columnwidth}\raggedright
Post-traumatic growth inventory\strut
\end{minipage} & \begin{minipage}[t]{0.22\columnwidth}\raggedright
An instrument for assessing positive outcomes reported by persons who have experienced traumatic events\strut
\end{minipage} & \begin{minipage}[t]{0.18\columnwidth}\raggedright
Tedeschi, \& Calhoun, 1996\strut
\end{minipage}\tabularnewline
\begin{minipage}[t]{0.22\columnwidth}\raggedright
usq\strut
\end{minipage} & \begin{minipage}[t]{0.27\columnwidth}\raggedright
Undergraduate stress questionnaire\strut
\end{minipage} & \begin{minipage}[t]{0.22\columnwidth}\raggedright
The undergraduate stress inventory presents students with various stressors and asks them to indicate if any of the events have happened to them. They are also asked how stressed they are by this event\strut
\end{minipage} & \begin{minipage}[t]{0.18\columnwidth}\raggedright
Crandall, Preisler, \& Aussprung, 1992\strut
\end{minipage}\tabularnewline
\begin{minipage}[t]{0.22\columnwidth}\raggedright
brcs\strut
\end{minipage} & \begin{minipage}[t]{0.27\columnwidth}\raggedright
Brief resilient coping scale\strut
\end{minipage} & \begin{minipage}[t]{0.22\columnwidth}\raggedright
The Brief Resilient Coping Scale (BRCS) is a 4-item measure designed to capture tendencies to cope with stress in a highly adaptive manner\strut
\end{minipage} & \begin{minipage}[t]{0.18\columnwidth}\raggedright
Sinclair \& Wallston, 2004\strut
\end{minipage}\tabularnewline
\bottomrule
\end{longtable}

\hypertarget{mental-health}{%
\subsection{Mental health}\label{mental-health}}

\begin{longtable}[]{@{}llll@{}}
\toprule
\begin{minipage}[b]{0.22\columnwidth}\raggedright
Title\strut
\end{minipage} & \begin{minipage}[b]{0.27\columnwidth}\raggedright
Name\strut
\end{minipage} & \begin{minipage}[b]{0.22\columnwidth}\raggedright
Description\strut
\end{minipage} & \begin{minipage}[b]{0.18\columnwidth}\raggedright
Reference\strut
\end{minipage}\tabularnewline
\midrule
\endhead
\begin{minipage}[t]{0.22\columnwidth}\raggedright
stai\strut
\end{minipage} & \begin{minipage}[t]{0.27\columnwidth}\raggedright
State-Trait Anxiety Inventory\strut
\end{minipage} & \begin{minipage}[t]{0.22\columnwidth}\raggedright
Measure of trait and state anxiety that can be used in clinical settings to diagnose anxiety and to distinguish it from depressive syndromes\strut
\end{minipage} & \begin{minipage}[t]{0.18\columnwidth}\raggedright
Spielberger, 1989\strut
\end{minipage}\tabularnewline
\begin{minipage}[t]{0.22\columnwidth}\raggedright
bdi\strut
\end{minipage} & \begin{minipage}[t]{0.27\columnwidth}\raggedright
Beck depression inventory\strut
\end{minipage} & \begin{minipage}[t]{0.22\columnwidth}\raggedright
Developed for the assessment of symptoms corresponding to criteria for diagnosing depressive disorders listed in the DSM IV\strut
\end{minipage} & \begin{minipage}[t]{0.18\columnwidth}\raggedright
Beck, Steer, \& Brown, 1996\strut
\end{minipage}\tabularnewline
\begin{minipage}[t]{0.22\columnwidth}\raggedright
mental\_health\_history\strut
\end{minipage} & \begin{minipage}[t]{0.27\columnwidth}\raggedright
Mental health history\strut
\end{minipage} & \begin{minipage}[t]{0.22\columnwidth}\raggedright
A questionnaire to assess mental health history\strut
\end{minipage} & \begin{minipage}[t]{0.18\columnwidth}\raggedright
Made by BABLab\strut
\end{minipage}\tabularnewline
\bottomrule
\end{longtable}

\hypertarget{social}{%
\subsection{Social}\label{social}}

\begin{longtable}[]{@{}llll@{}}
\toprule
\begin{minipage}[b]{0.22\columnwidth}\raggedright
Title\strut
\end{minipage} & \begin{minipage}[b]{0.27\columnwidth}\raggedright
Name\strut
\end{minipage} & \begin{minipage}[b]{0.22\columnwidth}\raggedright
Description\strut
\end{minipage} & \begin{minipage}[b]{0.18\columnwidth}\raggedright
Reference\strut
\end{minipage}\tabularnewline
\midrule
\endhead
\begin{minipage}[t]{0.22\columnwidth}\raggedright
uclals\strut
\end{minipage} & \begin{minipage}[t]{0.27\columnwidth}\raggedright
UCLA loneliness scale\strut
\end{minipage} & \begin{minipage}[t]{0.22\columnwidth}\raggedright
A 20-item scale designed to measure one's subjective feelings of loneliness as well as feelings of social isolation\strut
\end{minipage} & \begin{minipage}[t]{0.18\columnwidth}\raggedright
Russell, Peplau, \& Ferguson, 1978\strut
\end{minipage}\tabularnewline
\begin{minipage}[t]{0.22\columnwidth}\raggedright
scq\strut
\end{minipage} & \begin{minipage}[t]{0.27\columnwidth}\raggedright
Social craving questionnaire\strut
\end{minipage} & \begin{minipage}[t]{0.22\columnwidth}\raggedright
A measure designed to address social cravings.\strut
\end{minipage} & \begin{minipage}[t]{0.18\columnwidth}\raggedright
Made by BABLab\strut
\end{minipage}\tabularnewline
\begin{minipage}[t]{0.22\columnwidth}\raggedright
asc\strut
\end{minipage} & \begin{minipage}[t]{0.27\columnwidth}\raggedright
Adolescent social connection and coping during COVID-19\strut
\end{minipage} & \begin{minipage}[t]{0.22\columnwidth}\raggedright
Designed to learn about the ways you connect with people, and how it makes you feel. This might be affected by the COVID-19 outbreak, especially when following physical distancing or shelter-in-place orders\strut
\end{minipage} & \begin{minipage}[t]{0.18\columnwidth}\raggedright
Pfeifer et al., 2020\strut
\end{minipage}\tabularnewline
\begin{minipage}[t]{0.22\columnwidth}\raggedright
mspss\strut
\end{minipage} & \begin{minipage}[t]{0.27\columnwidth}\raggedright
Multidimensional scale of perceived social support\strut
\end{minipage} & \begin{minipage}[t]{0.22\columnwidth}\raggedright
Measures perceptions of support from 3 sources: Family, Friends, and a Significant Other\strut
\end{minipage} & \begin{minipage}[t]{0.18\columnwidth}\raggedright
Zimet, 1990\strut
\end{minipage}\tabularnewline
\begin{minipage}[t]{0.22\columnwidth}\raggedright
scs\strut
\end{minipage} & \begin{minipage}[t]{0.27\columnwidth}\raggedright
Social comparison scale\strut
\end{minipage} & \begin{minipage}[t]{0.22\columnwidth}\raggedright
Measures self-perceptions of social rank and relative social standing\strut
\end{minipage} & \begin{minipage}[t]{0.18\columnwidth}\raggedright
Allan \& Gilbert, 1995\strut
\end{minipage}\tabularnewline
\bottomrule
\end{longtable}

\hypertarget{personality}{%
\subsection{Personality}\label{personality}}

\begin{longtable}[]{@{}llll@{}}
\toprule
\begin{minipage}[b]{0.22\columnwidth}\raggedright
Title\strut
\end{minipage} & \begin{minipage}[b]{0.27\columnwidth}\raggedright
Name\strut
\end{minipage} & \begin{minipage}[b]{0.22\columnwidth}\raggedright
Description\strut
\end{minipage} & \begin{minipage}[b]{0.18\columnwidth}\raggedright
Reference\strut
\end{minipage}\tabularnewline
\midrule
\endhead
\begin{minipage}[t]{0.22\columnwidth}\raggedright
bfi\_10\strut
\end{minipage} & \begin{minipage}[t]{0.27\columnwidth}\raggedright
Big five personality inventory\strut
\end{minipage} & \begin{minipage}[t]{0.22\columnwidth}\raggedright
Inventory that measures an individual on the big five factors of personality (extraversion, agreeableness, conscentiousness, neuroticism, and openness to experience)\strut
\end{minipage} & \begin{minipage}[t]{0.18\columnwidth}\raggedright
Rammstedt \& John, 2007\strut
\end{minipage}\tabularnewline
\begin{minipage}[t]{0.22\columnwidth}\raggedright
ius\strut
\end{minipage} & \begin{minipage}[t]{0.27\columnwidth}\raggedright
Intolerance of uncertainty scale\strut
\end{minipage} & \begin{minipage}[t]{0.22\columnwidth}\raggedright
Examines self-reported degree of agreement with the idea that uncertainty is unacceptable, reflects badly on a person, and leads to frustration, stress, and the inability to take action\strut
\end{minipage} & \begin{minipage}[t]{0.18\columnwidth}\raggedright
Carleton, Norton, \& Asmundson, 2007\strut
\end{minipage}\tabularnewline
\bottomrule
\end{longtable}

\hypertarget{lifestyle}{%
\subsection{Lifestyle}\label{lifestyle}}

\begin{longtable}[]{@{}llll@{}}
\toprule
\begin{minipage}[b]{0.22\columnwidth}\raggedright
Title\strut
\end{minipage} & \begin{minipage}[b]{0.27\columnwidth}\raggedright
Name\strut
\end{minipage} & \begin{minipage}[b]{0.22\columnwidth}\raggedright
Description\strut
\end{minipage} & \begin{minipage}[b]{0.18\columnwidth}\raggedright
Reference\strut
\end{minipage}\tabularnewline
\midrule
\endhead
\begin{minipage}[t]{0.22\columnwidth}\raggedright
psqi\strut
\end{minipage} & \begin{minipage}[t]{0.27\columnwidth}\raggedright
Pittsburgh sleep quality index\strut
\end{minipage} & \begin{minipage}[t]{0.22\columnwidth}\raggedright
Measures the quality and patterns of sleep in adults, differentiating ``poor'' from ``good'' sleep quality by measuring seven areas (components): subjective sleep quality, sleep latency, sleep duration, habitual sleep efficiency, sleep disturbances, use of sleeping medications, and daytime dysfunction over the last month\strut
\end{minipage} & \begin{minipage}[t]{0.18\columnwidth}\raggedright
Buysse et al., 1989\strut
\end{minipage}\tabularnewline
\begin{minipage}[t]{0.22\columnwidth}\raggedright
timeline\strut
\end{minipage} & \begin{minipage}[t]{0.27\columnwidth}\raggedright
Timeline questionnaire\strut
\end{minipage} & \begin{minipage}[t]{0.22\columnwidth}\raggedright
Free response assessment of how participants spend an average day in hour increments\strut
\end{minipage} & \begin{minipage}[t]{0.18\columnwidth}\raggedright
Made by BABLab\strut
\end{minipage}\tabularnewline
\begin{minipage}[t]{0.22\columnwidth}\raggedright
bfq\strut
\end{minipage} & \begin{minipage}[t]{0.27\columnwidth}\raggedright
Brief food questionnaire\strut
\end{minipage} & \begin{minipage}[t]{0.22\columnwidth}\raggedright
Assesses participants consumption of food groups\strut
\end{minipage} & \begin{minipage}[t]{0.18\columnwidth}\raggedright
Made by BABLab\strut
\end{minipage}\tabularnewline
\begin{minipage}[t]{0.22\columnwidth}\raggedright
ipaq\strut
\end{minipage} & \begin{minipage}[t]{0.27\columnwidth}\raggedright
International physical activity questionnaire\strut
\end{minipage} & \begin{minipage}[t]{0.22\columnwidth}\raggedright
Asks questions on physical activity\strut
\end{minipage} & \begin{minipage}[t]{0.18\columnwidth}\raggedright
Booth, 2000\strut
\end{minipage}\tabularnewline
\bottomrule
\end{longtable}

\hypertarget{media}{%
\subsection{Media}\label{media}}

\begin{longtable}[]{@{}llll@{}}
\toprule
\begin{minipage}[b]{0.22\columnwidth}\raggedright
Title\strut
\end{minipage} & \begin{minipage}[b]{0.27\columnwidth}\raggedright
Name\strut
\end{minipage} & \begin{minipage}[b]{0.22\columnwidth}\raggedright
Description\strut
\end{minipage} & \begin{minipage}[b]{0.18\columnwidth}\raggedright
Reference\strut
\end{minipage}\tabularnewline
\midrule
\endhead
\begin{minipage}[t]{0.22\columnwidth}\raggedright
media\_consumption\strut
\end{minipage} & \begin{minipage}[t]{0.27\columnwidth}\raggedright
Media consumption questionnaire\strut
\end{minipage} & \begin{minipage}[t]{0.22\columnwidth}\raggedright
Examines media consumption in the typical two-week period before COVID-19 and current consumption. Social media usage, news consumption, public health information consumption, and COVID-19 beliefs are assessed\strut
\end{minipage} & \begin{minipage}[t]{0.18\columnwidth}\raggedright
Made by BABLab\strut
\end{minipage}\tabularnewline
\begin{minipage}[t]{0.22\columnwidth}\raggedright
smcs\strut
\end{minipage} & \begin{minipage}[t]{0.27\columnwidth}\raggedright
Social media craving scale\strut
\end{minipage} & \begin{minipage}[t]{0.22\columnwidth}\raggedright
Evaluates social media desire by examining frequency and duration of thoughts about social media use as well as degree of difficulty in resisting social media use\strut
\end{minipage} & \begin{minipage}[t]{0.18\columnwidth}\raggedright
Savci \& Griffiths, 2019\strut
\end{minipage}\tabularnewline
\bottomrule
\end{longtable}

\hypertarget{well-being}{%
\subsection{Well-being}\label{well-being}}

\begin{longtable}[]{@{}llll@{}}
\toprule
\begin{minipage}[b]{0.22\columnwidth}\raggedright
Title\strut
\end{minipage} & \begin{minipage}[b]{0.27\columnwidth}\raggedright
Name\strut
\end{minipage} & \begin{minipage}[b]{0.22\columnwidth}\raggedright
Description\strut
\end{minipage} & \begin{minipage}[b]{0.18\columnwidth}\raggedright
Reference\strut
\end{minipage}\tabularnewline
\midrule
\endhead
\begin{minipage}[t]{0.22\columnwidth}\raggedright
shs\strut
\end{minipage} & \begin{minipage}[t]{0.27\columnwidth}\raggedright
Subjective happiness scale\strut
\end{minipage} & \begin{minipage}[t]{0.22\columnwidth}\raggedright
A 4-item measure of gloval subjective happiness\strut
\end{minipage} & \begin{minipage}[t]{0.18\columnwidth}\raggedright
Lyubomirsky \& Lepper, 1999\strut
\end{minipage}\tabularnewline
\bottomrule
\end{longtable}

\hypertarget{qualitative}{%
\subsection{Qualitative}\label{qualitative}}

\begin{longtable}[]{@{}llll@{}}
\toprule
\begin{minipage}[b]{0.22\columnwidth}\raggedright
Title\strut
\end{minipage} & \begin{minipage}[b]{0.27\columnwidth}\raggedright
Name\strut
\end{minipage} & \begin{minipage}[b]{0.22\columnwidth}\raggedright
Description\strut
\end{minipage} & \begin{minipage}[b]{0.18\columnwidth}\raggedright
Reference\strut
\end{minipage}\tabularnewline
\midrule
\endhead
\begin{minipage}[t]{0.22\columnwidth}\raggedright
written\_response\strut
\end{minipage} & \begin{minipage}[t]{0.27\columnwidth}\raggedright
Long form qualitative written response\strut
\end{minipage} & \begin{minipage}[t]{0.22\columnwidth}\raggedright
Prompts respondent to reflect on and describe within a 5 minute time frame how COVID-19 has impacted daily life\strut
\end{minipage} & \begin{minipage}[t]{0.18\columnwidth}\raggedright
Adapted from Pennebaker, 1997\strut
\end{minipage}\tabularnewline
\bottomrule
\end{longtable}

\textbf{Prompt:}

\emph{I would like for you to write about your very deepest thoughts and feelings about the way COVID-19 has affected you and your life. I'd like you to really let go and explore your very deepest emotions and thoughts. You might tie your topic to your relationships with others including parents, lovers, friends, or relatives, to your past, your present, of your future, or to who you have been, who you would like to be, or who you are now. All of your writing will be completely confidential. Don't worry about spelling, sentence structure or grammar. The only rule is that you begin writing and keep writing until 5 minutes have passed.}

\hypertarget{procedure}{%
\section{Procedure}\label{procedure}}

\hypertarget{recruitment}{%
\subsection{Recruitment}\label{recruitment}}

Participants were recruited in two ways - via SONA and via a raffle.

The following materials were used:

\begin{figure}
\centering
\includegraphics{images/inside_out_instagram.png}
\caption{}
\end{figure}

\hypertarget{timing}{%
\subsection{Timing}\label{timing}}

Pilot time:

\begin{itemize}
\tightlist
\item
  Research Assistant \#1 - 1 hour and 50 minutes
\item
  Research Assistant \#2 - 1 hour and 15 minutes
\item
  Research Assistant \#3 - 1 hour and 5 minutes
\end{itemize}

\hypertarget{questionnaire-order}{%
\subsection{Questionnaire Order}\label{questionnaire-order}}

\begin{enumerate}
\def\labelenumi{\arabic{enumi}.}
\tightlist
\item
  panas (assessed 3 times - once at beginning, once before writing, once after writing)
\item
  information
\item
  somna
\item
  covid\_objective
\item
  somatic\_symptoms (assessed currently and retrospectively before COVID-19)
\item
  pss
\item
  hai (assessed currently and retrospectively before COVID-19)
\item
  bdi\_ii (assessed currently and retrospectively before COVID-19)
\item
  pill
\item
  covid\_subjective
\item
  pedsql\_gi (assessed currently and retrospectively before COVID-19)
\item
  media\_consumption
\item
  ctq
\item
  sci
\item
  psqi
\item
  cte
\item
  timeline
\item
  uclals
\item
  sasrq
\item
  ccfq
\item
  maia (assessed currently and retrospectively before COVID-19)
\item
  stai
\item
  usq
\item
  bfq (assessed currently and retrospectively before COVID-19)
\item
  asc
\item
  demographics
\item
  shs
\item
  mspss
\item
  ipaq (assessed currently and retrospectively before COVID-19)
\item
  ius
\item
  smcs
\item
  bfi
\item
  ptgi\_brcs
\item
  mental\_health\_history
\item
  med\_check
\item
  gastro
\item
  rome
\item
  menstrual
\item
  panas
\item
  written\_response
\item
  panas
\end{enumerate}

\hypertarget{attention-checks}{%
\subsection{Attention Checks}\label{attention-checks}}

\begin{enumerate}
\def\labelenumi{\arabic{enumi}.}
\tightlist
\item
  covid\_objective
\item
  covid\_subjective
\item
  media\_consumption
\item
  scq
\item
  sasrq
\item
  bfq
\item
  ipaq
\item
  ptgi
\end{enumerate}

\hypertarget{study-2---egg---methods}{%
\chapter{Study 2 - EGG - Methods}\label{study-2---egg---methods}}

\hypertarget{information-1}{%
\section{Information}\label{information-1}}

\hypertarget{enrollment}{%
\subsection{Enrollment}\label{enrollment}}

Participants will sign up for lab session slots at least 48 hours ahead of time to ensure ability to consent and complete home session. We will set the SONA to reflect the 48 hour rule as well as instruct participants that they must complete the online session before their scheduled lab session.

\hypertarget{consent}{%
\subsection{Consent}\label{consent}}

The researcher will email the participant the consent form to review and sign electronically. Any questions will be answered at this time. We will direct female participants to wear a sports bra and a comfortable t-shirt. We will mention that we have a hospital gown if they prefer. We will also mention that it might be a male experimenter, so for all participants to look at the diagram and be sure they are comfortable (female experimenter can be available if necessary). We will also ask male participants if they are comfortable shaving any chest/stomach hair for the session and will send them a photo demonstrating the regions involved. We will also tell participants that their response time will be monitored, so to please read and respond carefully to all surveys.

\hypertarget{home-session}{%
\subsection{Home Session}\label{home-session}}

\begin{itemize}
\tightlist
\item
  Assign the participant a Participant ID
\item
  Log the ID, name, email address, and contact information (if given) into the ID drive
\item
  Find the participants previous REDCap enrollment or enroll the participant on REDCap
\item
  Email the participant (using the email template) the home session REDCap link, including instructions for the home and lab session.
\end{itemize}

\hypertarget{lab-session}{%
\subsection{Lab Session}\label{lab-session}}

Before the lab session, the researcher will check the home session submissions and record the time to completion to ensure that responses are being paid attention to, the researcher will also monitor the attention check items.

\begin{itemize}
\item
  The researcher will book the Bear's Den using the calendar system for the scheduled timeslots.
\item
  All lab sessions take place between 12 p.m. and 6 p.m. Participants are instructed not to eat for 1 hour before their session.
\item
  Participants will complete several questionnaires in the lab including the med\_check, female health, panas, stai\_state, and info.
\item
  A two part SONA study will require the participant to complete the home session before the lab session.
\end{itemize}

\hypertarget{procedure-1}{%
\section{Procedure}\label{procedure-1}}

\textbf{Participants will:}

\begin{enumerate}
\def\labelenumi{\arabic{enumi}.}
\tightlist
\item
  Drink bottle of water
\item
  Consent
\item
  Apply physiology stickers
\item
  Round 1 of movie watching
\item
  Complete PANAS
\item
  Round 2 of movie watching
\item
  Complete PANAS
\item
  Round 3 of movie watching
\item
  Complete PANAS
\item
  Height, weight, and waist measurement
\item
  Debrief
\end{enumerate}

\hypertarget{file-structure}{%
\subsection{File Structure}\label{file-structure}}

Each participant's file should include

\begin{itemize}
\tightlist
\item
  3 physio files
\item
  3 psychopy files
\item
  1 photocopy of the run sheet
\end{itemize}

\hypertarget{run-sheet}{%
\subsection{Run Sheet}\label{run-sheet}}

\textbf{Will include:}

\begin{itemize}
\tightlist
\item
  Observer ratings of startle scale
\item
  Notes on laughing nervously, jumping, screaming, shrieking (to account for movement)
\end{itemize}

\hypertarget{references}{%
\chapter{References}\label{references}}

Allan, S. \& Gilbert, P. (1995). A social comparison scale: Psychometric properties and
relationship to psychopathology. Personality and Individual Differences, 19, 293-299.

Beck, A. T., Steer, R. A., \& Brown, G. K. (1996). Beck depression inventory-II. San Antonio, 78(2), 490-498.

Bernstein DP, Fink L, Handelsman L, Lovejoy M, Wenzel K, Sapareto E, Gurriero J. (1994). Initial reliability and
validity of a new retrospective measure of child abuse and neglect. Am J Psychiatry; 151: 1132-
1136

Booth, M.L. (2000). Assessment of Physical Activity: An International Perspective. Research Quarterly for Exercise and Sport, 71 (2): s114-20.

Buysse, D.J., Reynolds III, C.F., Monk, T.H., Berman, S.R., \& Kupfer, D.J. (1989). The Pittsburgh Sleep Quality Index: A new instrument for psychiatric practice and research. Journal of Psychiatric Research, 28(2), 193-213.

Callaghan, B. L., Fields, A., Gee, D. G., Gabard-Durnam, L., Caldera, C., Humphreys, K. L., Goff, B., Flannery, J., Telzer, E. H., Shapiro, M., \& Tottenham, N. (2020). Mind and gut: Associations between mood and gastrointestinal distress in children exposed to adversity. Development and Psychopathology, 32(1), 309--328. \url{https://doi.org/10.1017/S0954579419000087}

Carleton, R. N., Norton, M. A. P. J., \& Asmundson, G. J. G. (2007). Fearing the unknown: A short version of the Intolerance of Uncertainty Scale. Journal of Anxiety Disorders, 21(1), 105--117. \url{https://doi.org/10.1016/j.janxdis.2006.03.014}

Cardeña, E., Koopman, C., Classen, C., Waelde, L. C., \& Spiegel, D. (2000). Psychometric properties of the Stanford Acute Stress Reaction Questionnaire (SASRQ): A valid and reliable measure of acute stress. Journal of Traumatic Stress, 13(4), 719--734. \url{https://doi.org/10.1023/A:1007822603186}

Cohen, S., Kamarck, T., and Mermelstein, R. (1983). A global measure of perceived stress. Journal of Health and Social Behavior, 24, 386-396.

Crandall, C.S., Preisler, J.J., \& Aussprung, J. (1992). Measuring life event stress in the lives of college students: The undergraduate stress questionnaire. Journal of Behavioral Medicine, 15, 627-662.

Firestein, M., \& Callaghan, B. (2019). The brain--gut connection: Environmental influences on gastrointestinal biology and neurobehavior across development. Developmental Psychobiology, 61(5), 639--639. \url{https://doi.org/10.1002/dev.21869}

Gabrys, R. L., Tabri, N., Anisman, H., \& Matheson, K. (2018). Cognitive Control and Flexibility in the Context of Stress and Depressive Symptoms: The Cognitive Control and Flexibility Questionnaire. Frontiers in Psychology, 9. \url{https://doi.org/10.3389/fpsyg.2018.02219}

John, O. P., \& Srivastava, S. (1999). The Big-Five trait taxonomy: History, measurement, and theoretical perspectives. In L. A. Pervin \& O. P. John (Eds.), Handbook of personality: Theory and research (Vol. 2, pp.~102--138). New York: Guilford Press.

Körber, S., Frieser, D., Steinbrecher, N., \& Hiller, W. (2011). Classification characteristics of the Patient Health Questionnaire-15 for screening somatoform disorders in a primary care setting. Journal of Psychosomatic Research, 71(3), 142--147. \url{https://doi.org/10.1016/j.jpsychores.2011.01.006}

Lee, S., Wu, J., Ma, Y. L., Tsang, A., Guo, W.-J., \& Sung, J. (2009). Irritable bowel syndrome is strongly associated with generalized anxiety disorder: A community study. Alimentary Pharmacology \& Therapeutics, 30(6), 643--651. \url{https://doi.org/10.1111/j.1365-2036.2009.04074}.

Lyubomirsky, S., \& Lepper, H. (1999). A measure of subjective happiness: Preliminary reliability and construct validation. Social Indicators Research, 46, 137-155.

Mehling, W. E., Price, C., Daubenmier, J. J., Acree, M., Bartmess, E., and Stewart, A. (2012). The multidimensional assessment of interoceptive awareness (MAIA). PLoS ONE 7:e48230. doi: 10.1371/journal.pone.0048230

Pennebaker, J. W. (1982). The Psychology of Physical Symptoms. New York: Springer-Verlag.

Pennebaker, J. W. (1997). Writing About Emotional Experiences as a Therapeutic Process. Psychological Science, 8(3), 162--166. \url{https://doi.org/10.1111/j.1467-9280.1997.tb00403.x}

Pennebaker, J.W. \& Susman, J.R. (2013) Childhood Trauma Questionnaire. Measurement Instrument Database for the Social Science. Retrieved from www.midss.ie

Russell, D , Peplau, L. A.. \& Ferguson, M. L. (1978). Developing a measure of loneliness.
Journal of Personality Assessment, 42, 290-294.

Salkovskis, M, Rimes, K.A., Warwick, H.M.C, Clark, D.M. The Health Anxiety Inventory: development and validation of scales for the measurement of health anxiety and hypochondriasis. Psychol Med. 2006; 32(5):843--853.

Savci, M., \& Griffiths, M. D. (2019). The Development of the Turkish Social Media Craving Scale (SMCS): A Validation Study. International Journal of Mental Health and Addiction. \url{https://doi.org/10.1007/s11469-019-00062-9}

Sinclair, V. G., \& Wallston, K.A. (2004). The development and psychometric evaluation of the Brief Resilient Coping Scale. Assessment, 11 (1), 94-101.

Spielberger, C. D., Gorsuch, R. L., Lushene, R., Vagg, P. R., \& Jacobs, G. A. (1983). Manual for the State-Trait Anxiety Inventory. Palo Alto, CA: Consulting Psychologists Press.

Steiner, M., Macdougall, M., \& Brown, E. (2003). The premenstrual symptoms screening tool (PSST) for clinicians. Archives of Women's Mental Health, 6(3), 203--209. \url{https://doi.org/10.1007/s00737-003-0018-4}

Tedeschi, R. G., \& Calhoun, L. G. (1996). The Posttraumatic Growth Inventory: Measuring the positive legacy of trauma. Journal of Traumatic Stress, 9(3), 455--471. \url{https://doi.org/10.1007/bf02103658}

Varni, J.W., Kay, M.T., Limbers, C.A., Franciosi, J.P., \& Pohl, J.F. (2012). PedsQL™ Gastrointestinal Symptoms Module item development: Qualitative methods. Journal of Pediatric Gastroenterology \& Nutrition, 54, 664-671

Vianna, E. P. M., \& Tranel, D. (2006). Gastric myoelectrical activity as an index of emotional arousal. International Journal of Psychophysiology, 61(1), 70--76. \url{https://doi.org/10.1016/j.ijpsycho.2005.10.019}

Zimet, G.D., Dahlem, N.W., Zimet, S.G. \& Farley, G.K. (1988). The Multidimensional Scale of
Perceived Social Support. Journal of Personality Assessment, 52, 30-41.

\end{document}
