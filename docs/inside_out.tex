\PassOptionsToPackage{unicode=true}{hyperref} % options for packages loaded elsewhere
\PassOptionsToPackage{hyphens}{url}
%
\documentclass[]{book}
\usepackage{lmodern}
\usepackage{amssymb,amsmath}
\usepackage{ifxetex,ifluatex}
\usepackage{fixltx2e} % provides \textsubscript
\ifnum 0\ifxetex 1\fi\ifluatex 1\fi=0 % if pdftex
  \usepackage[T1]{fontenc}
  \usepackage[utf8]{inputenc}
  \usepackage{textcomp} % provides euro and other symbols
\else % if luatex or xelatex
  \usepackage{unicode-math}
  \defaultfontfeatures{Ligatures=TeX,Scale=MatchLowercase}
\fi
% use upquote if available, for straight quotes in verbatim environments
\IfFileExists{upquote.sty}{\usepackage{upquote}}{}
% use microtype if available
\IfFileExists{microtype.sty}{%
\usepackage[]{microtype}
\UseMicrotypeSet[protrusion]{basicmath} % disable protrusion for tt fonts
}{}
\IfFileExists{parskip.sty}{%
\usepackage{parskip}
}{% else
\setlength{\parindent}{0pt}
\setlength{\parskip}{6pt plus 2pt minus 1pt}
}
\usepackage{hyperref}
\hypersetup{
            pdftitle={Inside Out},
            pdfauthor={Emily Towner},
            pdfborder={0 0 0},
            breaklinks=true}
\urlstyle{same}  % don't use monospace font for urls
\usepackage{longtable,booktabs}
% Fix footnotes in tables (requires footnote package)
\IfFileExists{footnote.sty}{\usepackage{footnote}\makesavenoteenv{longtable}}{}
\usepackage{graphicx,grffile}
\makeatletter
\def\maxwidth{\ifdim\Gin@nat@width>\linewidth\linewidth\else\Gin@nat@width\fi}
\def\maxheight{\ifdim\Gin@nat@height>\textheight\textheight\else\Gin@nat@height\fi}
\makeatother
% Scale images if necessary, so that they will not overflow the page
% margins by default, and it is still possible to overwrite the defaults
% using explicit options in \includegraphics[width, height, ...]{}
\setkeys{Gin}{width=\maxwidth,height=\maxheight,keepaspectratio}
\setlength{\emergencystretch}{3em}  % prevent overfull lines
\providecommand{\tightlist}{%
  \setlength{\itemsep}{0pt}\setlength{\parskip}{0pt}}
\setcounter{secnumdepth}{5}
% Redefines (sub)paragraphs to behave more like sections
\ifx\paragraph\undefined\else
\let\oldparagraph\paragraph
\renewcommand{\paragraph}[1]{\oldparagraph{#1}\mbox{}}
\fi
\ifx\subparagraph\undefined\else
\let\oldsubparagraph\subparagraph
\renewcommand{\subparagraph}[1]{\oldsubparagraph{#1}\mbox{}}
\fi

% set default figure placement to htbp
\makeatletter
\def\fps@figure{htbp}
\makeatother

\usepackage{booktabs}
\usepackage{amsthm}
\makeatletter
\def\thm@space@setup{%
  \thm@preskip=8pt plus 2pt minus 4pt
  \thm@postskip=\thm@preskip
}
\makeatother
\usepackage[]{natbib}
\bibliographystyle{apalike}

\title{Inside Out}
\author{Emily Towner}
\date{2020-05-05}

\begin{document}
\maketitle

{
\setcounter{tocdepth}{1}
\tableofcontents
}
\hypertarget{introduction}{%
\chapter{Introduction}\label{introduction}}

The EGG and Emotionality

\hypertarget{study-1---covid---about}{%
\chapter{Study 1 - COVID - About}\label{study-1---covid---about}}

\hypertarget{summary}{%
\section{Summary}\label{summary}}

TEST

This study will explore the relationship between the global COVID-19 pandemic, somatic symptoms, and psychological stress.

This study will explore the relationship between emotions and somatic symptamology in the wake of the COVID-19 pandemic, while exploring individual differences in social, environmental, personality, and lifestyle factors which may mitigate or exacerbate the negative psychological impact of this stressor.

The specific aims of this study are to (1) establish a relationship between psychological stress and somatic symptoms as assessed by our newly developed Somatic Symptoms of Negative Affect (somna) questionnaire, (2) investigate the individual differences that might influence the somatic and psychological response to stress (such as early life stress, social support, media consumption, diet and exercise, lifestyle habits, trait variables, etc.), and (3) examine how specific somatic symptoms in the context of stress may relate to mental health.

We will recruit N = 200 (minimum 150) adult participants. Participants will be recruited online from the UCLA participant pool in the university term directly after the outbreak of COVID-19. Participants will complete a range of questionnaires assessing current levels of stress, social and emotional support and functioning, physical health symptoms, early life adversity, media exposure and consumption, and lifestyle factors.

\hypertarget{keywords}{%
\section{Keywords}\label{keywords}}

stress, emotions, mental health, somatic symptoms

\hypertarget{background}{%
\section{Background}\label{background}}

While we often describe our emotions as ``gut feelings'', surprisingly little research has examined how emotions and the gastrointestinal system interact. Given the onset of a global pandemic, the situation provides a unique opportunity to investigate how an emotion inducing real-world event, COVID-19, might influence somatic symptoms and the stress response.

Prior research during public health crises such as the SARS epidemic in 2006 reveal that the stress associated with quarantine during the epidemic was associated with higher symptoms of acute stress disorder and later post traumatic stress symptoms \citep{baiSurveyStressReactions2004}. However, research also reveals that in the wake of the SARS epidemic, individuals found their friends and family members more supportive \citep{lauPositiveMentalHealthrelated2006}. Similarly, research suggests that social support obtained through social interactions after the events of September 11th, 2001, reduced college students' symptoms of both depression and physical illness \citep{macgeorgeStressSocialSupport2004}.

Given that gastrointestinal and mental heath problems are highly comorbid, with anxiety five times higher in individuals with irritable bowel syndrome (IBS) than in those with no IBS symptoms (Lee et al., 2009), gastrointestinal and somatic symptoms may serve as a useful indicator of emotional functioning, particularly during this period of heightened awareness of physical health amid the COVID-19 pandemic. For example, stress in early life can affect both emotional and gastrointestinal symptoms and functioning. One study demonstrated that previous adverse care experiences were associated with both increased anxiety and incidence of gastrointestinal symptoms in youth (Callaghan et al., 2019). In addition, early adversity was associated with changes in gastrointestinal microbiome diversity that were correlated with neural activation to emotional faces (Callaghan et al., 2019).

Physiological methods such as heart rate and sweat response are common indicators of emotional arousal, but the electrograstrogram (EGG) is seldom used in psychological research. In one study, researchers found that movie clips capturing emotions of fear, disgust, and sadness, elicited a greater EGG response relative to a neutral condition (Vianna \& Tranel, 2006).

Few studies have explored the way that individual differences including early stress, social support, media consumption, and lifestyle factors may mitigate or exacerbate the negative somatic psychological impact of a stressor as well as the way that these variables and emotional functioning may influence the EGG response to negatively emotionally arousing movie clips.

Using qustionnaires and electrogastrography, we seek to investigate the relationship between somatic symptoms (particularly gastrointestinal symptoms) gastric myoelectrical activity, and emotional functioning, in the context of a public health crisis as well as during emotionally arousing movie clips. We also hope to explore factors that might influence gastrointestinal responses to emotional arousal, and whether and how physical sensations are associated with emotions and physiological responses. We also hope to evaluate how EGG sits with other physiological measures, such as heart rate and sweat response, in order to explore whether emotional patterning of physiological responses contribute to meaningful differences in emotion regulation, the stress response, and mental health.

\hypertarget{specific-aims}{%
\section{Specific Aims}\label{specific-aims}}

The study will test several hypotheses.

\begin{enumerate}
\def\labelenumi{\arabic{enumi}.}
\tightlist
\item
  Establish a relationship between COVID-19 stress and somatic symptoms

  \begin{itemize}
  \tightlist
  \item
    Higher perceived stress during the outbreak of COVID-19 will be associated with greater somatic symptomology on the somna.
  \item
    Increased interoceptive awareness since the onset of COVID-19 will be associated with greater somatic symptomology
  \item
    Increased health anxiety since the onset of COVID-19 will be associated with greater somatic symtpomology
  \end{itemize}
\item
  Examine how specific somatic symptoms in the context of COVID-19 stress may relate to mental health.

  \begin{itemize}
  \tightlist
  \item
    Somatic symptomology will mediate the relationship between perceived stress and anxiety, depression, and panic.
  \end{itemize}
\item
  Investigate individual differences that might influence the emotional response to COVID-19 stress (such as early life stress, social support, media consumption, diet and exercise, lifestyle habits, trait variables, etc.).

  \begin{itemize}
  \tightlist
  \item
    Early life stress will be associated with an increased emotional response to COVID-19 stress.
  \item
    Social support, lifestyle habits (such as sleep, diet, and exercise), personality traits, and media consumption will moderate the association between current stress and emotional response.
  \end{itemize}
\end{enumerate}

\hypertarget{study-1---covid---methods}{%
\chapter{Study 1 - COVID - Methods}\label{study-1---covid---methods}}

\hypertarget{measures}{%
\section{Measures}\label{measures}}

\hypertarget{information}{%
\subsection{Information}\label{information}}

\begin{longtable}[]{@{}llll@{}}
\toprule
\begin{minipage}[b]{0.22\columnwidth}\raggedright
Title\strut
\end{minipage} & \begin{minipage}[b]{0.27\columnwidth}\raggedright
Name\strut
\end{minipage} & \begin{minipage}[b]{0.22\columnwidth}\raggedright
Description\strut
\end{minipage} & \begin{minipage}[b]{0.18\columnwidth}\raggedright
Reference\strut
\end{minipage}\tabularnewline
\midrule
\endhead
\begin{minipage}[t]{0.22\columnwidth}\raggedright
info\strut
\end{minipage} & \begin{minipage}[t]{0.27\columnwidth}\raggedright
Information questionnaire\strut
\end{minipage} & \begin{minipage}[t]{0.22\columnwidth}\raggedright
Assesses demographics, health, and location information\strut
\end{minipage} & \begin{minipage}[t]{0.18\columnwidth}\raggedright
Made by BABLab\strut
\end{minipage}\tabularnewline
\begin{minipage}[t]{0.22\columnwidth}\raggedright
covid\_objective\strut
\end{minipage} & \begin{minipage}[t]{0.27\columnwidth}\raggedright
Objective impact of COVID-19\strut
\end{minipage} & \begin{minipage}[t]{0.22\columnwidth}\raggedright
Assesses the objective impact of COVID-19 including infection, quarantine, household, social distancing etc.\strut
\end{minipage} & \begin{minipage}[t]{0.18\columnwidth}\raggedright
Made by BABLab\strut
\end{minipage}\tabularnewline
\bottomrule
\end{longtable}

\hypertarget{somatic}{%
\subsection{Somatic}\label{somatic}}

\begin{longtable}[]{@{}llll@{}}
\toprule
\begin{minipage}[b]{0.22\columnwidth}\raggedright
Title\strut
\end{minipage} & \begin{minipage}[b]{0.27\columnwidth}\raggedright
Name\strut
\end{minipage} & \begin{minipage}[b]{0.22\columnwidth}\raggedright
Description\strut
\end{minipage} & \begin{minipage}[b]{0.18\columnwidth}\raggedright
Reference\strut
\end{minipage}\tabularnewline
\midrule
\endhead
\begin{minipage}[t]{0.22\columnwidth}\raggedright
somna\strut
\end{minipage} & \begin{minipage}[t]{0.27\columnwidth}\raggedright
Somatic markers of negative affect\strut
\end{minipage} & \begin{minipage}[t]{0.22\columnwidth}\raggedright
Assesses physical sensations of anxiety and sadness, where they are located, and their intensity\strut
\end{minipage} & \begin{minipage}[t]{0.18\columnwidth}\raggedright
Made by BABLab\strut
\end{minipage}\tabularnewline
\begin{minipage}[t]{0.22\columnwidth}\raggedright
maia\strut
\end{minipage} & \begin{minipage}[t]{0.27\columnwidth}\raggedright
Multidimensional Assessment of Interoceptive Awareness\strut
\end{minipage} & \begin{minipage}[t]{0.22\columnwidth}\raggedright
Multidimensional self-report measure of interoceptive body awareness\strut
\end{minipage} & \begin{minipage}[t]{0.18\columnwidth}\raggedright
\strut
\end{minipage}\tabularnewline
\begin{minipage}[t]{0.22\columnwidth}\raggedright
hai\strut
\end{minipage} & \begin{minipage}[t]{0.27\columnwidth}\raggedright
Health anxiety inventory\strut
\end{minipage} & \begin{minipage}[t]{0.22\columnwidth}\raggedright
Assesses people's anxiety about health symptoms (hypochondriasis)\strut
\end{minipage} & \begin{minipage}[t]{0.18\columnwidth}\raggedright
\strut
\end{minipage}\tabularnewline
\begin{minipage}[t]{0.22\columnwidth}\raggedright
ss\strut
\end{minipage} & \begin{minipage}[t]{0.27\columnwidth}\raggedright
Somatic symptoms\strut
\end{minipage} & \begin{minipage}[t]{0.22\columnwidth}\raggedright
Assesses a range of somatic symptoms in adult participants\strut
\end{minipage} & \begin{minipage}[t]{0.18\columnwidth}\raggedright
\strut
\end{minipage}\tabularnewline
\begin{minipage}[t]{0.22\columnwidth}\raggedright
pill\strut
\end{minipage} & \begin{minipage}[t]{0.27\columnwidth}\raggedright
Pennebaker inventory of limbid languidness\strut
\end{minipage} & \begin{minipage}[t]{0.22\columnwidth}\raggedright
Measures people's tendency to notice and report a braod array of physical symptoms and sensations.\strut
\end{minipage} & \begin{minipage}[t]{0.18\columnwidth}\raggedright
\strut
\end{minipage}\tabularnewline
\begin{minipage}[t]{0.22\columnwidth}\raggedright
pedsql\_gi\strut
\end{minipage} & \begin{minipage}[t]{0.27\columnwidth}\raggedright
Pediatric Quality of life -- Gastrointestinal Symptoms Module\strut
\end{minipage} & \begin{minipage}[t]{0.22\columnwidth}\raggedright
Assess incidence of Gastrointestinal Symptoms and fatigue in children\strut
\end{minipage} & \begin{minipage}[t]{0.18\columnwidth}\raggedright
\strut
\end{minipage}\tabularnewline
\begin{minipage}[t]{0.22\columnwidth}\raggedright
med\_check\strut
\end{minipage} & \begin{minipage}[t]{0.27\columnwidth}\raggedright
Medication checklist\strut
\end{minipage} & \begin{minipage}[t]{0.22\columnwidth}\raggedright
List of medications that participants are on -- needed for physiology analyses as well as verification of physical health issues.\strut
\end{minipage} & \begin{minipage}[t]{0.18\columnwidth}\raggedright
\strut
\end{minipage}\tabularnewline
\begin{minipage}[t]{0.22\columnwidth}\raggedright
gastrointestinal\_disorders\strut
\end{minipage} & \begin{minipage}[t]{0.27\columnwidth}\raggedright
Gastrointetsinal disorders questionnaire\strut
\end{minipage} & \begin{minipage}[t]{0.22\columnwidth}\raggedright
Assesses gastrointestinal issues, their frequency and intensity\strut
\end{minipage} & \begin{minipage}[t]{0.18\columnwidth}\raggedright
\strut
\end{minipage}\tabularnewline
\begin{minipage}[t]{0.22\columnwidth}\raggedright
rome\strut
\end{minipage} & \begin{minipage}[t]{0.27\columnwidth}\raggedright
Rome IV criteria questionnaire\strut
\end{minipage} & \begin{minipage}[t]{0.22\columnwidth}\raggedright
Assesses the presence of symptoms which meet criteria for irritable bowel syndrome as stated by the Rome IV\strut
\end{minipage} & \begin{minipage}[t]{0.18\columnwidth}\raggedright
\strut
\end{minipage}\tabularnewline
\begin{minipage}[t]{0.22\columnwidth}\raggedright
menstrual\_cycle\strut
\end{minipage} & \begin{minipage}[t]{0.27\columnwidth}\raggedright
Menstrual cycle questionnaire\strut
\end{minipage} & \begin{minipage}[t]{0.22\columnwidth}\raggedright
Assesses menstrual phase, which affects gastrointestinal responding, as well as medications which may affect menstrual phase such as oral contraceptive use\strut
\end{minipage} & \begin{minipage}[t]{0.18\columnwidth}\raggedright
\strut
\end{minipage}\tabularnewline
\begin{minipage}[t]{0.22\columnwidth}\raggedright
psst\strut
\end{minipage} & \begin{minipage}[t]{0.27\columnwidth}\raggedright
Premenstrual symptoms screening tool\strut
\end{minipage} & \begin{minipage}[t]{0.22\columnwidth}\raggedright
Assesses premenstrual syndromes and criteria for premenstrual dysphoric disorder (pmdd) as well as premenstrual syndrome (pms).\strut
\end{minipage} & \begin{minipage}[t]{0.18\columnwidth}\raggedright
\strut
\end{minipage}\tabularnewline
\bottomrule
\end{longtable}

\hypertarget{stress}{%
\subsection{Stress}\label{stress}}

\begin{longtable}[]{@{}llll@{}}
\toprule
\begin{minipage}[b]{0.22\columnwidth}\raggedright
Title\strut
\end{minipage} & \begin{minipage}[b]{0.27\columnwidth}\raggedright
Name\strut
\end{minipage} & \begin{minipage}[b]{0.22\columnwidth}\raggedright
Description\strut
\end{minipage} & \begin{minipage}[b]{0.18\columnwidth}\raggedright
Reference\strut
\end{minipage}\tabularnewline
\midrule
\endhead
\begin{minipage}[t]{0.22\columnwidth}\raggedright
covid\_subjective\strut
\end{minipage} & \begin{minipage}[t]{0.27\columnwidth}\raggedright
Subjective impact of COVID-19\strut
\end{minipage} & \begin{minipage}[t]{0.22\columnwidth}\raggedright
Assesses the subjective impact of COVID-19 on well-being.\strut
\end{minipage} & \begin{minipage}[t]{0.18\columnwidth}\raggedright
\strut
\end{minipage}\tabularnewline
\begin{minipage}[t]{0.22\columnwidth}\raggedright
pss\strut
\end{minipage} & \begin{minipage}[t]{0.27\columnwidth}\raggedright
Perceived stress scale\strut
\end{minipage} & \begin{minipage}[t]{0.22\columnwidth}\raggedright
The Perceived Stress Scale is a classic stress assesment instrument. This tool, while originally developed in 1983, remains a popular choice for helping us understand how different situations affect our feelings and our perceived stress. The questions in this scale as about your feelings and thoughts during the last month. In each case, you willl be asked ot indicate how often.\strut
\end{minipage} & \begin{minipage}[t]{0.18\columnwidth}\raggedright
\strut
\end{minipage}\tabularnewline
\begin{minipage}[t]{0.22\columnwidth}\raggedright
sasrq\strut
\end{minipage} & \begin{minipage}[t]{0.27\columnwidth}\raggedright
Stanford acute stress reaction questionnaire\strut
\end{minipage} & \begin{minipage}[t]{0.22\columnwidth}\raggedright
Assesses the psychological symptoms experienced in the aftermath of a traumatic event.\strut
\end{minipage} & \begin{minipage}[t]{0.18\columnwidth}\raggedright
\strut
\end{minipage}\tabularnewline
\begin{minipage}[t]{0.22\columnwidth}\raggedright
cte\strut
\end{minipage} & \begin{minipage}[t]{0.27\columnwidth}\raggedright
Childhood traumatic events questionnaire\strut
\end{minipage} & \begin{minipage}[t]{0.22\columnwidth}\raggedright
The Childhood Traumautic Events Questionnaire is a brief survey of six early traumatic experiences (death, divorce, violence, sexual abuse, illnesss, and upheaval).\strut
\end{minipage} & \begin{minipage}[t]{0.18\columnwidth}\raggedright
\strut
\end{minipage}\tabularnewline
\begin{minipage}[t]{0.22\columnwidth}\raggedright
ctq\strut
\end{minipage} & \begin{minipage}[t]{0.27\columnwidth}\raggedright
Childhood trauma questionnaire\strut
\end{minipage} & \begin{minipage}[t]{0.22\columnwidth}\raggedright
The Childhood Trauma Questionnaire (CTQ) is a self-report instrument covering 28 items, to rate the severity of emotional abuse and neglect, physical abuse and neglect and sexual abuse. It has been validated in terms of psychometric test properties in samples of psychiatric patients, i.e.~drug and substance abusers. This data set includes five CTQ subscale scores.\strut
\end{minipage} & \begin{minipage}[t]{0.18\columnwidth}\raggedright
\strut
\end{minipage}\tabularnewline
\begin{minipage}[t]{0.22\columnwidth}\raggedright
ccfq\strut
\end{minipage} & \begin{minipage}[t]{0.27\columnwidth}\raggedright
Cognitive control and flexibility questionnaire\strut
\end{minipage} & \begin{minipage}[t]{0.22\columnwidth}\raggedright
Measures an individual's percevied ability to exert control over intrusive, unwanted (negative) thoughts and emotions, and their ability to flexibly cope with a stressful situation.\strut
\end{minipage} & \begin{minipage}[t]{0.18\columnwidth}\raggedright
\strut
\end{minipage}\tabularnewline
\begin{minipage}[t]{0.22\columnwidth}\raggedright
ptgi\strut
\end{minipage} & \begin{minipage}[t]{0.27\columnwidth}\raggedright
Post-traumatic growth inventory\strut
\end{minipage} & \begin{minipage}[t]{0.22\columnwidth}\raggedright
An instrument for assessing positive outcomes reported by persons who have experienced traumatic events.\strut
\end{minipage} & \begin{minipage}[t]{0.18\columnwidth}\raggedright
\strut
\end{minipage}\tabularnewline
\begin{minipage}[t]{0.22\columnwidth}\raggedright
usq\strut
\end{minipage} & \begin{minipage}[t]{0.27\columnwidth}\raggedright
Undergraduate stress questionnaire\strut
\end{minipage} & \begin{minipage}[t]{0.22\columnwidth}\raggedright
The undergraduate stress inventory presents students with various stressors and asks them to indicate if any of the events have happened to them. They are also asked how stressed they are by this event.\strut
\end{minipage} & \begin{minipage}[t]{0.18\columnwidth}\raggedright
\strut
\end{minipage}\tabularnewline
\begin{minipage}[t]{0.22\columnwidth}\raggedright
brcs\strut
\end{minipage} & \begin{minipage}[t]{0.27\columnwidth}\raggedright
Brief resilient coping scale\strut
\end{minipage} & \begin{minipage}[t]{0.22\columnwidth}\raggedright
The Brief Resilient Coping Scale (BRCS) is a 4-item measure designed to capture tendencies to cope with stress in a highly adaptive manner\strut
\end{minipage} & \begin{minipage}[t]{0.18\columnwidth}\raggedright
\strut
\end{minipage}\tabularnewline
\bottomrule
\end{longtable}

\hypertarget{mental-health}{%
\subsection{Mental health}\label{mental-health}}

\begin{longtable}[]{@{}llll@{}}
\toprule
\begin{minipage}[b]{0.22\columnwidth}\raggedright
Title\strut
\end{minipage} & \begin{minipage}[b]{0.27\columnwidth}\raggedright
Name\strut
\end{minipage} & \begin{minipage}[b]{0.22\columnwidth}\raggedright
Description\strut
\end{minipage} & \begin{minipage}[b]{0.18\columnwidth}\raggedright
Reference\strut
\end{minipage}\tabularnewline
\midrule
\endhead
\begin{minipage}[t]{0.22\columnwidth}\raggedright
stai\strut
\end{minipage} & \begin{minipage}[t]{0.27\columnwidth}\raggedright
State-Trait Anxiety Inventory\strut
\end{minipage} & \begin{minipage}[t]{0.22\columnwidth}\raggedright
The State-Trait Anxiety Inventory (STAI) is a commonly used measure of trait and state anxiety. It can be used in clinical settings to diagnose anxiety and to distinguish it from depressive syndromes.\strut
\end{minipage} & \begin{minipage}[t]{0.18\columnwidth}\raggedright
\strut
\end{minipage}\tabularnewline
\begin{minipage}[t]{0.22\columnwidth}\raggedright
bdi\strut
\end{minipage} & \begin{minipage}[t]{0.27\columnwidth}\raggedright
Beck depression inventory\strut
\end{minipage} & \begin{minipage}[t]{0.22\columnwidth}\raggedright
Developed for the assessment of symptoms corresponding to criteria for diagnosing depressive disorders listed in the DSM IV\strut
\end{minipage} & \begin{minipage}[t]{0.18\columnwidth}\raggedright
\strut
\end{minipage}\tabularnewline
\begin{minipage}[t]{0.22\columnwidth}\raggedright
mental\_health\_history\strut
\end{minipage} & \begin{minipage}[t]{0.27\columnwidth}\raggedright
Mental health history\strut
\end{minipage} & \begin{minipage}[t]{0.22\columnwidth}\raggedright
A questionnaire to assess mental health history\strut
\end{minipage} & \begin{minipage}[t]{0.18\columnwidth}\raggedright
\strut
\end{minipage}\tabularnewline
\bottomrule
\end{longtable}

\hypertarget{social}{%
\subsection{Social}\label{social}}

\begin{longtable}[]{@{}llll@{}}
\toprule
\begin{minipage}[b]{0.22\columnwidth}\raggedright
Title\strut
\end{minipage} & \begin{minipage}[b]{0.27\columnwidth}\raggedright
Name\strut
\end{minipage} & \begin{minipage}[b]{0.22\columnwidth}\raggedright
Description\strut
\end{minipage} & \begin{minipage}[b]{0.18\columnwidth}\raggedright
Reference\strut
\end{minipage}\tabularnewline
\midrule
\endhead
\begin{minipage}[t]{0.22\columnwidth}\raggedright
uclals\strut
\end{minipage} & \begin{minipage}[t]{0.27\columnwidth}\raggedright
UCLA loneliness scale\strut
\end{minipage} & \begin{minipage}[t]{0.22\columnwidth}\raggedright
A 20-item scale designed to measure one's subjective feelings of loneliness as well as feelings of social isolation.\strut
\end{minipage} & \begin{minipage}[t]{0.18\columnwidth}\raggedright
\strut
\end{minipage}\tabularnewline
\begin{minipage}[t]{0.22\columnwidth}\raggedright
scq\strut
\end{minipage} & \begin{minipage}[t]{0.27\columnwidth}\raggedright
Social craving questionnaire\strut
\end{minipage} & \begin{minipage}[t]{0.22\columnwidth}\raggedright
A measure designed to address social cravings.\strut
\end{minipage} & \begin{minipage}[t]{0.18\columnwidth}\raggedright
Made by BABLab\strut
\end{minipage}\tabularnewline
\begin{minipage}[t]{0.22\columnwidth}\raggedright
asc\strut
\end{minipage} & \begin{minipage}[t]{0.27\columnwidth}\raggedright
Adolescent social connection and coping during COVID-19\strut
\end{minipage} & \begin{minipage}[t]{0.22\columnwidth}\raggedright
This questionnaire is designed to learn about the ways you connect with people, and how it makes you feel. This might be affected by the COVID-19 outbreak, especially when following physical distancing or shelter-in-place orders.\strut
\end{minipage} & \begin{minipage}[t]{0.18\columnwidth}\raggedright
\strut
\end{minipage}\tabularnewline
\begin{minipage}[t]{0.22\columnwidth}\raggedright
mspss\strut
\end{minipage} & \begin{minipage}[t]{0.27\columnwidth}\raggedright
Multidimensional scale of perceived social support\strut
\end{minipage} & \begin{minipage}[t]{0.22\columnwidth}\raggedright
The Multidimensional Scale of Perceived Social Support (MSPSS) is a brief research tool designed to measure perceptions of support from 3 sources: Family, Friends, and a Significant Other. The scale is comprised of a total of 12 items, with 4 items for each subscale.\strut
\end{minipage} & \begin{minipage}[t]{0.18\columnwidth}\raggedright
\strut
\end{minipage}\tabularnewline
\begin{minipage}[t]{0.22\columnwidth}\raggedright
scs\strut
\end{minipage} & \begin{minipage}[t]{0.27\columnwidth}\raggedright
Social comparison scale\strut
\end{minipage} & \begin{minipage}[t]{0.22\columnwidth}\raggedright
This scale was developed by Allan and Gilbert (1995) to measure self-perceptions of social rank and relative social standing. This scale uses a semantic differential methodology and consists of 11 bipolar constructs.\strut
\end{minipage} & \begin{minipage}[t]{0.18\columnwidth}\raggedright
\strut
\end{minipage}\tabularnewline
\bottomrule
\end{longtable}

\hypertarget{personality}{%
\subsection{Personality}\label{personality}}

\begin{longtable}[]{@{}llll@{}}
\toprule
\begin{minipage}[b]{0.22\columnwidth}\raggedright
Title\strut
\end{minipage} & \begin{minipage}[b]{0.27\columnwidth}\raggedright
Name\strut
\end{minipage} & \begin{minipage}[b]{0.22\columnwidth}\raggedright
Description\strut
\end{minipage} & \begin{minipage}[b]{0.18\columnwidth}\raggedright
Reference\strut
\end{minipage}\tabularnewline
\midrule
\endhead
\begin{minipage}[t]{0.22\columnwidth}\raggedright
bfi\_10\strut
\end{minipage} & \begin{minipage}[t]{0.27\columnwidth}\raggedright
Big five personality inventory\strut
\end{minipage} & \begin{minipage}[t]{0.22\columnwidth}\raggedright
Inventory that measures an individual on the big five factors of personality (extraversion, agreeableness, conscentiousness, neuroticism, and openness to experience).\strut
\end{minipage} & \begin{minipage}[t]{0.18\columnwidth}\raggedright
\strut
\end{minipage}\tabularnewline
\begin{minipage}[t]{0.22\columnwidth}\raggedright
ius\strut
\end{minipage} & \begin{minipage}[t]{0.27\columnwidth}\raggedright
Intolerance of uncertainty scale\strut
\end{minipage} & \begin{minipage}[t]{0.22\columnwidth}\raggedright
The Intolerance of Uncertainty Scale includes items relating to the idea that uncertainty is unacceptable, reflects badly on a person, and leads to frustration, stress, and the inability to take action\strut
\end{minipage} & \begin{minipage}[t]{0.18\columnwidth}\raggedright
\strut
\end{minipage}\tabularnewline
\bottomrule
\end{longtable}

\hypertarget{lifestyle}{%
\subsection{Lifestyle}\label{lifestyle}}

\begin{longtable}[]{@{}llll@{}}
\toprule
Title & Name & Description & Reference\tabularnewline
\midrule
\endhead
\bottomrule
\end{longtable}

\begin{itemize}
\tightlist
\item
  Sleep
\item
  Timeline
\item
  Diet (brief food questionnaire)
\item
  Exercise (adapt the international physical activity questionnaire - pre and during)
\item
  Productivity
\end{itemize}

\hypertarget{media}{%
\subsection{Media}\label{media}}

\begin{itemize}
\tightlist
\item
  Social media use
\item
  Traditional media consumption
\item
  Public health information consumption
\item
  Use of technology/new media to socialize
\item
  Screen time usage
\item
  smcs
\end{itemize}

\hypertarget{well-being}{%
\subsection{Well-being}\label{well-being}}

\begin{itemize}
\tightlist
\item
  shs
\end{itemize}

\hypertarget{qualitative}{%
\subsection{Qualitative}\label{qualitative}}

\begin{itemize}
\tightlist
\item
  Long form qualitative written response (adapted from Pennebaker, 1997)(5-10 minutes timed)
\end{itemize}

I would like for you to write about your very deepest thoughts and feelings about the way COVID-19 has affected you and your life. I'd like you to really let go and explore your very deepest emotions and thoughts. You might tie your topic to your relationships with others including parents, lovers, friends, or relatives, to your past, your present, of your future, or to who you have been, who you would like to be, or who you are now. All of your writing will be completely confidential. Don't worry about spelling, sentence structure or grammar. The only rule is that you begin writing and keep writing until 5 minutes have passed.

\hypertarget{procedure}{%
\section{Procedure}\label{procedure}}

\hypertarget{timing}{%
\subsection{Timing}\label{timing}}

Pilot time:

\begin{itemize}
\tightlist
\item
  Nicole - 1 hour and 50 minutes
\item
  Danielle - 1 hour and 15 minutes
\item
  Chloe - 1 hour and 5 minutes
\end{itemize}

\hypertarget{questionnaire-order}{%
\subsection{Questionnaire Order}\label{questionnaire-order}}

\begin{enumerate}
\def\labelenumi{\arabic{enumi}.}
\tightlist
\item
  panas (assessed 3 times - once at beginning, once before writing, once after writing)
\item
  information
\item
  somna
\item
  covid\_objective
\item
  somatic\_symptoms (assessed currently and retrospectively before COVID-19)
\item
  pss
\item
  hai (assessed currently and retrospectively before COVID-19)
\item
  bdi\_ii (assessed currently and retrospectively before COVID-19)
\item
  pill
\item
  covid\_subjective
\item
  pedsql\_gi (assessed currently and retrospectively before COVID-19)
\item
  media\_consumption
\item
  ctq
\item
  sci
\item
  psqi
\item
  cte
\item
  timeline
\item
  uclals
\item
  sasrq
\item
  ccfq
\item
  maia (assessed currently and retrospectively before COVID-19)
\item
  stai
\item
  usq
\item
  bfq (assessed currently and retrospectively before COVID-19)
\item
  asc
\item
  demographics
\item
  shs
\item
  mspss
\item
  ipaq (assessed currently and retrospectively before COVID-19)
\item
  ius
\item
  smcs
\item
  bfi
\item
  ptgi\_brcs
\item
  mental\_health\_history
\item
  med\_check
\item
  gastro
\item
  rome
\item
  menstrual
\item
  panas
\item
  written\_response
\item
  panas
\end{enumerate}

\hypertarget{attention-checks}{%
\subsection{Attention Checks}\label{attention-checks}}

\begin{enumerate}
\def\labelenumi{\arabic{enumi}.}
\tightlist
\item
  covid\_objective
\item
  covid\_subjective
\item
  media\_consumption
\item
  scq
\item
  sasrq
\item
  bfq
\item
  ipaq
\item
  ptgi
\end{enumerate}

\hypertarget{study-2---egg---about}{%
\chapter{Study 2 - EGG - About}\label{study-2---egg---about}}

\hypertarget{summary-1}{%
\section{Summary}\label{summary-1}}

This study will explore the relationship between gastrointestinal activity and emotions utilizing electrogastrography. The specific aims of this study are to (1) establish a relationship between emotionally arousing stimuli and the EGG response, (2) investigate the individual differences that might influence the EGG response to stress (such as early life stress, current stress, trait variables, etc.), (3) examine how the EGG response sits with other physiological indices (such as heart rate and sweat response), and (4) explore the ways in which physical sensations are associated with emotions and physiological responses.

We will recruit N = 200 (minimum 150) adult participants. Participants will watch a series of sad, scary, and neutral movies while electrophysiology recordings are made. Then they will complete a range of questionnaires assessing social and emotional functioning, physical health symptoms, early life adversity, and physical health assessments.

\hypertarget{keywords-1}{%
\section{Keywords}\label{keywords-1}}

egg, stress, emotions, physiology

\hypertarget{background-1}{%
\section{Background}\label{background-1}}

While we often describe our emotions as ``gut feelings'', surprisingly little research has examined how emotions and the gastrointestinal system interact. Physiological methods such as heart rate and sweat response are common indicators of emotional arousal, but the electrograstrogram (EGG) is seldom used in psychological research. Given that gastrointestinal and mental heath problems are highly comorbid, with anxiety five times higher in individuals with irritable bowel syndrome (IBS) than in those with no IBS symptoms (Lee et al., 2009), gastrointestinal activity may serve as a useful indicator of emotional functioning. In one study, researchers found that movie clips capturing emotions of fear, disgust, and sadness, elicited a greater EGG response relative to a neutral condition (Vianna \& Tranel, 2006).

Interestingly, few studies have explored the way that individual differences, such as anxiety, depression, stress, and early adversity might influence the EGG response to these negatively emotionally arousing movie clips. For example, stress in early life can affect both emotional and gastrointestinal symptoms and functioning. One study demonstrated that previous adverse care experiences were associated with both increased anxiety and incidence of gastrointestinal symptoms in youth (Callaghan et al., 2019). In addition, early adversity was associated with changes in gastrointestinal microbiome diversity that were correlated with neural activation to emotional faces (Callaghan et al., 2019).

Using electrogastrography, we seek to investigate the relationship between gastric myoelectrical activity and emotionally arousing movie clips. We also hope to explore factors that might influence gastrointestinal responses to emotional arousal, and whether and how physical sensations are associated with emotions and physiological responses. We also hope to evaluate how EGG sits with other physiological measures, such as heart rate and sweat response, in order to explore whether emotional patterning of physiological responses contribute to meaningful differences in emotion regulation, the stress response, and mental health.

\hypertarget{specific-aims-1}{%
\section{Specific Aims}\label{specific-aims-1}}

The study will test several hypotheses.

\begin{enumerate}
\def\labelenumi{\arabic{enumi}.}
\tightlist
\item
  Establish a relationship between emotionally arousing stimuli and the EGG response.

  \begin{itemize}
  \tightlist
  \item
    There will be a greater EGG response (i.e.~average peak amplitude) for the sad and scary movie condition relative to the neutral movie condition.
  \item
    The intensity of subjective emotion will be positively correlated with EGG response.
  \end{itemize}
\item
  Investigate the individual differences that might influence the EGG response to stress (such as early life stress, current stress, trait variables, etc.)

  \begin{itemize}
  \tightlist
  \item
    Increased levels of emotional distress, such as anxiety and depression, will be associated with greater EGG response during the emotionally arousing movie conditions.
  \item
    Early life stress will be associated with greater EGG response in the emotionally arousing movie conditions, relative to individuals who did not experience early life stress.
  \item
    Greater current and perceived stress will be associated with greater EGG response in the emotionally arousing conditions.
  \end{itemize}
\item
  Examine how the EGG response sits with other physiological indices (such as heart rate and sweat response)

  \begin{itemize}
  \tightlist
  \item
    The EGG response will be associated with other physiological indices of emotional arousal, such as heart rate and sweat response.
  \end{itemize}
\item
  Explore the ways in which physical sensations are associated with emotions and physiological responses

  \begin{itemize}
  \tightlist
  \item
    Lower interoceptive awareness will be associated with greater physiological responses to emotionally arousing stimuli.
  \item
    Higher somatic symptomology will be associated with greater physiological responses to emotionally arousing stimuli.
  \item
    Distinct dimensions of physical sensations and physiological responding will be positively correlated.
  \item
    Gastrointestinal symptoms will be associated with both greater anxiety and greater EGG response to emotionally arousing stimuli.
  \end{itemize}
\end{enumerate}

\hypertarget{study-1---covid---methods-1}{%
\chapter{Study 1 - COVID - Methods}\label{study-1---covid---methods-1}}

\hypertarget{information-1}{%
\section{Information}\label{information-1}}

Enrollment
Participants will sign up for lab session slots at least 48 hours ahead of time to ensure ability to consent and complete home session. We will set the SONA to reflect the 48 hour rule as well as instruct participants that they must complete the online session before their scheduled lab session.

Consent
The researcher will email the participant the consent form to review and sign electronically. Any questions will be answered at this time. We will direct female participants to wear a sports bra and a comfortable t-shirt. We will mention that we have a hospital gown if they prefer. We will also mention that it might be a male experimenter, so for all participants to look at the diagram and be sure they are comfortable (female experimenter can be available if necessary). We will also ask male participants if they are comfortable shaving any chest/stomach hair for the session and will send them a photo demonstrating the regions involved. We will also tell participants that their response time will be monitored, so to please read and respond carefully to all surveys.

Home Session
Assign the participant a Participant ID
Log the ID, name, email address, and contact information (if given) into the ID drive
Enroll the participant on REDCap
Email the participant (using the email template) the home session REDCap link, including instructions for the home and lab session.
Lab Session
Before the lab session, the researcher will check the home session submissions and record the time to completion to ensure that responses are being paid attention to.

All lab sessions take place between 12 p.m. and 6 p.m. Participants are instructed not to eat for 1 hour before their session.

Participants will complete several questionnaires in the lab including the med\_check, female health, panas, stai\_state, and info.

A two part SONA study will require the participant to complete the home session before the lab session.

The researcher will book the Bear's Den using the calendar system.

Drink bottle of water
Consent
Apply physiology stickers
Round 1 of movie watching
Complete PANAS
Round 2 of movie watching
Complete PANAS
Round 3 of movie watching
Complete PANAS
Height, weight, and waist measurement
Debrief
Files
Each participant file should include

3 physio files
3 psychopy files
1 photocopy of the run sheet
Run Sheet
Will include

Observer ratings of startle scale
Notes on laughing nervously, jumping, screaming, shrieking (to account for movement)
Data Analysis
Will include factor analysis for somna

\hypertarget{references}{%
\chapter{References}\label{references}}

Callaghan, B. L., Fields, A., Gee, D. G., Gabard-Durnam, L., Caldera, C., Humphreys, K. L., Goff, B., Flannery, J., Telzer, E. H., Shapiro, M., \& Tottenham, N. (2020). Mind and gut: Associations between mood and gastrointestinal distress in children exposed to adversity. Development and Psychopathology, 32(1), 309--328. \url{https://doi.org/10.1017/S0954579419000087}

Firestein, M., \& Callaghan, B. (2019). The brain--gut connection: Environmental influences on gastrointestinal biology and neurobehavior across development. Developmental Psychobiology, 61(5), 639--639. \url{https://doi.org/10.1002/dev.21869}

Lee, S., Wu, J., Ma, Y. L., Tsang, A., Guo, W.-J., \& Sung, J. (2009). Irritable bowel syndrome is strongly associated with generalized anxiety disorder: A community study. Alimentary Pharmacology \& Therapeutics, 30(6), 643--651. \url{https://doi.org/10.1111/j.1365-2036.2009.04074.x}

Vianna, E. P. M., \& Tranel, D. (2006). Gastric myoelectrical activity as an index of emotional arousal. International Journal of Psychophysiology, 61(1), 70--76. \url{https://doi.org/10.1016/j.ijpsycho.2005.10.019}

\bibliography{book.bib,packages.bib,ref.bib}

\end{document}
