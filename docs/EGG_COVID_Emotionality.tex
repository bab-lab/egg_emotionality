\documentclass[]{book}
\usepackage{lmodern}
\usepackage{amssymb,amsmath}
\usepackage{ifxetex,ifluatex}
\usepackage{fixltx2e} % provides \textsubscript
\ifnum 0\ifxetex 1\fi\ifluatex 1\fi=0 % if pdftex
  \usepackage[T1]{fontenc}
  \usepackage[utf8]{inputenc}
\else % if luatex or xelatex
  \ifxetex
    \usepackage{mathspec}
  \else
    \usepackage{fontspec}
  \fi
  \defaultfontfeatures{Ligatures=TeX,Scale=MatchLowercase}
\fi
% use upquote if available, for straight quotes in verbatim environments
\IfFileExists{upquote.sty}{\usepackage{upquote}}{}
% use microtype if available
\IfFileExists{microtype.sty}{%
\usepackage{microtype}
\UseMicrotypeSet[protrusion]{basicmath} % disable protrusion for tt fonts
}{}
\usepackage{hyperref}
\hypersetup{unicode=true,
            pdftitle={EGG/COVID-19 and Emotionality},
            pdfauthor={Emily Towner},
            pdfborder={0 0 0},
            breaklinks=true}
\urlstyle{same}  % don't use monospace font for urls
\usepackage{natbib}
\bibliographystyle{apalike}
\usepackage{longtable,booktabs}
\usepackage{graphicx,grffile}
\makeatletter
\def\maxwidth{\ifdim\Gin@nat@width>\linewidth\linewidth\else\Gin@nat@width\fi}
\def\maxheight{\ifdim\Gin@nat@height>\textheight\textheight\else\Gin@nat@height\fi}
\makeatother
% Scale images if necessary, so that they will not overflow the page
% margins by default, and it is still possible to overwrite the defaults
% using explicit options in \includegraphics[width, height, ...]{}
\setkeys{Gin}{width=\maxwidth,height=\maxheight,keepaspectratio}
\IfFileExists{parskip.sty}{%
\usepackage{parskip}
}{% else
\setlength{\parindent}{0pt}
\setlength{\parskip}{6pt plus 2pt minus 1pt}
}
\setlength{\emergencystretch}{3em}  % prevent overfull lines
\providecommand{\tightlist}{%
  \setlength{\itemsep}{0pt}\setlength{\parskip}{0pt}}
\setcounter{secnumdepth}{5}
% Redefines (sub)paragraphs to behave more like sections
\ifx\paragraph\undefined\else
\let\oldparagraph\paragraph
\renewcommand{\paragraph}[1]{\oldparagraph{#1}\mbox{}}
\fi
\ifx\subparagraph\undefined\else
\let\oldsubparagraph\subparagraph
\renewcommand{\subparagraph}[1]{\oldsubparagraph{#1}\mbox{}}
\fi

%%% Use protect on footnotes to avoid problems with footnotes in titles
\let\rmarkdownfootnote\footnote%
\def\footnote{\protect\rmarkdownfootnote}

%%% Change title format to be more compact
\usepackage{titling}

% Create subtitle command for use in maketitle
\providecommand{\subtitle}[1]{
  \posttitle{
    \begin{center}\large#1\end{center}
    }
}

\setlength{\droptitle}{-2em}

  \title{EGG/COVID-19 and Emotionality}
    \pretitle{\vspace{\droptitle}\centering\huge}
  \posttitle{\par}
    \author{Emily Towner}
    \preauthor{\centering\large\emph}
  \postauthor{\par}
      \predate{\centering\large\emph}
  \postdate{\par}
    \date{2020-04-01}

\usepackage{booktabs}
\usepackage{amsthm}
\makeatletter
\def\thm@space@setup{%
  \thm@preskip=8pt plus 2pt minus 4pt
  \thm@postskip=\thm@preskip
}
\makeatother

\begin{document}
\maketitle

{
\setcounter{tocdepth}{1}
\tableofcontents
}
\hypertarget{introduction}{%
\chapter{Introduction}\label{introduction}}

The EGG and Emotionality

\hypertarget{study-1---covid---about}{%
\chapter{Study 1 - COVID - About}\label{study-1---covid---about}}

\hypertarget{summary}{%
\section{Summary}\label{summary}}

This study will explore the relationship between the global COVID-19 pandemic, somatic symptoms, and psychological stress.

This study will explore the relationship between emotions and somatic symptamology in the wake of the COVID-19 pandemic, while exploring individual differences in social, environmental, personality, and lifestyle factors which may mitigate or exacerbate the negative psychological impact of this stressor.

The specific aims of this study are to (1) establish a relationship between psychological stress and somatic symptoms as assessed by our newly developed Somatic Symptoms of Negative Affect (somna) questionnaire, (2) investigate the individual differences that might influence the somatic and psychological response to stress (such as early life stress, social support, media consumption, diet and exercise, lifestyle habits, trait variables, etc.), and (3) examine how specific somatic symptoms in the context of stress may relate to mental health.

We will recruit N = 200 (minimum 150) adult participants. Participants will be recruited online from the UCLA participant pool in the university term directly after the outbreak of COVID-19. Participants will complete a range of questionnaires assessing current levels of stress, social and emotional support and functioning, physical health symptoms, early life adversity, media exposure and consumption, and lifestyle factors.

\hypertarget{keywords}{%
\section{Keywords}\label{keywords}}

stress, emotions, mental health, somatic symptoms

\hypertarget{background}{%
\section{Background}\label{background}}

\hypertarget{specific-aims}{%
\section{Specific Aims}\label{specific-aims}}

The study will test several hypotheses.

\begin{enumerate}
\def\labelenumi{\arabic{enumi}.}
\tightlist
\item
  Establish a relationship between COVID-19 stress and somatic symptoms

  \begin{itemize}
  \tightlist
  \item
    Higher perceived stress during the outbreak of COVID-19 will be associated with greater somatic symptomology on the somna.
  \item
    Increased interoceptive awareness since the onset of COVID-19 will be associated with greater somatic symptomology
  \item
    Increased health anxiety since the onset of COVID-19 will be associated with greater somatic symtpomology
  \end{itemize}
\item
  Examine how specific somatic symptoms in the context of COVID-19 stress may relate to mental health.

  \begin{itemize}
  \tightlist
  \item
    Somatic symptomology will mediate the relationship between perceived stress and anxiety, depression, and panic.
  \end{itemize}
\item
  Investigate individual differences that might influence the emotional response to COVID-19 stress (such as early life stress, social support, media consumption, diet and exercise, lifestyle habits, trait variables, etc.).

  \begin{itemize}
  \tightlist
  \item
    Early life stress will be associated with an increased emotional response to COVID-19 stress.
  \item
    Social support, lifestyle habits (such as sleep, diet, and exercise), personality traits, and media consumption will moderate the association between current stress and emotional response.
  \end{itemize}
\end{enumerate}

\hypertarget{study-1---covid---methods}{%
\chapter{Study 1 - COVID - Methods}\label{study-1---covid---methods}}

\hypertarget{measures}{%
\section{Measures}\label{measures}}

\hypertarget{demographics}{%
\subsection{Demographics}\label{demographics}}

\begin{itemize}
\tightlist
\item
  Questionnaire to assess

  \begin{itemize}
  \tightlist
  \item
    SES
  \item
    Work situation (or job loss)
  \item
    Financial situation (or financial stress)
  \item
    Home country
  \end{itemize}
\end{itemize}

\hypertarget{somatic}{%
\subsection{Somatic}\label{somatic}}

\begin{itemize}
\tightlist
\item
  Somatic markers of negative affect (somna)
\item
  Interoceptive awareness (pre-COVID and during)(maia)
\item
  Health anxiety (pre-COVID and during)
\item
  Somatic symptoms (pre-COVID and during)
\item
  Pennebaker Inventory of Limbid Languidness (pill)

  \begin{itemize}
  \tightlist
  \item
    Add don't know (don't pay attention to) response
  \end{itemize}
\item
  Pedsql\_gi
\item
  Medication checklist
\item
  Gastrointestinal disorders
\item
  Rome
\end{itemize}

\hypertarget{stress}{%
\subsection{Stress}\label{stress}}

\begin{itemize}
\tightlist
\item
  Objective impact (know anyone who has it, how impacted is community, social distancing measures in place)
\item
  Perceived stress scale (pss)
\item
  Impact of event scale (ies)
\item
  Early life stress (ctq)
\item
  Cognitive capacity (ability to concentrate etc.)
\item
  Mental health history
\item
  Post-traumatic growth
\item
  Acute stress (stanford acute stress reaction questionnaire)
\end{itemize}

\hypertarget{mental-health}{%
\subsection{Mental health}\label{mental-health}}

\begin{itemize}
\tightlist
\item
  DSM-5 Self-Rated Level 1 Cross-Cutting Symptom Measure

  \begin{itemize}
  \tightlist
  \item
    I think maybe we should skip logic this to include the other level 2 dsm questionnaires from online
  \end{itemize}
\item
  State and trait anxiety inventory (stai)
\item
  Depression (bdi\_ii)
\item
  Mental health history
\end{itemize}

\hypertarget{social}{%
\subsection{Social}\label{social}}

\begin{itemize}
\tightlist
\item
  Objective social support

  \begin{itemize}
  \tightlist
  \item
    How many people isolating with
  \item
    What social networks are in place
  \item
    How often are they communicating with others
  \item
    Other received social support
  \end{itemize}
\item
  Perceived social support (pss)
\item
  UCLA loneliness scale
\item
  Social craving (may need to adapt substance craving scale)
\end{itemize}

\hypertarget{personality}{%
\subsection{Personality}\label{personality}}

\begin{itemize}
\tightlist
\item
  Introversion and extroversion scale
\item
  Tolerance for uncertainty/ambiguity/change
\end{itemize}

\hypertarget{lifestyle}{%
\subsection{Lifestyle}\label{lifestyle}}

\begin{itemize}
\tightlist
\item
  Sleep
\item
  Timeline
\item
  Diet
\item
  Exercise
\item
  Productivity
\end{itemize}

\hypertarget{media}{%
\subsection{Media}\label{media}}

\begin{itemize}
\tightlist
\item
  Social media use
\item
  Traditional media consumption
\item
  Public health information consumption
\item
  Use of technology/new media to socialize
\end{itemize}

\hypertarget{qualitative}{%
\subsection{Qualitative}\label{qualitative}}

\begin{itemize}
\tightlist
\item
  Long form qualitative written response
\end{itemize}

\hypertarget{study-2---egg---about}{%
\chapter{Study 2 - EGG - About}\label{study-2---egg---about}}

\hypertarget{summary-1}{%
\section{Summary}\label{summary-1}}

This study will explore the relationship between gastrointestinal activity and emotions utilizing electrogastrography. The specific aims of this study are to (1) establish a relationship between emotionally arousing stimuli and the EGG response, (2) investigate the individual differences that might influence the EGG response to stress (such as early life stress, current stress, trait variables, etc.), (3) examine how the EGG response sits with other physiological indices (such as heart rate and sweat response), and (4) explore the ways in which physical sensations are associated with emotions and physiological responses.

We will recruit N = 200 (minimum 150) adult participants. Participants will watch a series of sad, scary, and neutral movies while electrophysiology recordings are made. Then they will complete a range of questionnaires assessing social and emotional functioning, physical health symptoms, early life adversity, and physical health assessments.

\hypertarget{keywords-1}{%
\section{Keywords}\label{keywords-1}}

egg, stress, emotions, physiology

\hypertarget{background-1}{%
\section{Background}\label{background-1}}

While we often describe our emotions as ``gut feelings'', surprisingly little research has examined how emotions and the gastrointestinal system interact. Physiological methods such as heart rate and sweat response are common indicators of emotional arousal, but the electrograstrogram (EGG) is seldom used in psychological research. Given that gastrointestinal and mental heath problems are highly comorbid, with anxiety five times higher in individuals with irritable bowel syndrome (IBS) than in those with no IBS symptoms (Lee et al., 2009), gastrointestinal activity may serve as a useful indicator of emotional functioning. In one study, researchers found that movie clips capturing emotions of fear, disgust, and sadness, elicited a greater EGG response relative to a neutral condition (Vianna \& Tranel, 2006).

Interestingly, few studies have explored the way that individual differences, such as anxiety, depression, stress, and early adversity might influence the EGG response to these negatively emotionally arousing movie clips. For example, stress in early life can affect both emotional and gastrointestinal symptoms and functioning. One study demonstrated that previous adverse care experiences were associated with both increased anxiety and incidence of gastrointestinal symptoms in youth (Callaghan et al., 2019). In addition, early adversity was associated with changes in gastrointestinal microbiome diversity that were correlated with neural activation to emotional faces (Callaghan et al., 2019).

Using electrogastrography, we seek to investigate the relationship between gastric myoelectrical activity and emotionally arousing movie clips. We also hope to explore factors that might influence gastrointestinal responses to emotional arousal, and whether and how physical sensations are associated with emotions and physiological responses. We also hope to evaluate how EGG sits with other physiological measures, such as heart rate and sweat response, in order to explore whether emotional patterning of physiological responses contribute to meaningful differences in emotion regulation, the stress response, and mental health.

\hypertarget{specific-aims-1}{%
\section{Specific Aims}\label{specific-aims-1}}

The study will test several hypotheses.

\begin{enumerate}
\def\labelenumi{\arabic{enumi}.}
\tightlist
\item
  Establish a relationship between emotionally arousing stimuli and the EGG response.

  \begin{itemize}
  \tightlist
  \item
    There will be a greater EGG response (i.e.~average peak amplitude) for the sad and scary movie condition relative to the neutral movie condition.
  \item
    The intensity of subjective emotion will be positively correlated with EGG response.
  \end{itemize}
\item
  Investigate the individual differences that might influence the EGG response to stress (such as early life stress, current stress, trait variables, etc.)

  \begin{itemize}
  \tightlist
  \item
    Increased levels of emotional distress, such as anxiety and depression, will be associated with greater EGG response during the emotionally arousing movie conditions.
  \item
    Early life stress will be associated with greater EGG response in the emotionally arousing movie conditions, relative to individuals who did not experience early life stress.
  \item
    Greater current and perceived stress will be associated with greater EGG response in the emotionally arousing conditions.
  \end{itemize}
\item
  Examine how the EGG response sits with other physiological indices (such as heart rate and sweat response)

  \begin{itemize}
  \tightlist
  \item
    The EGG response will be associated with other physiological indices of emotional arousal, such as heart rate and sweat response.
  \end{itemize}
\item
  Explore the ways in which physical sensations are associated with emotions and physiological responses

  \begin{itemize}
  \tightlist
  \item
    Lower interoceptive awareness will be associated with greater physiological responses to emotionally arousing stimuli.
  \item
    Higher somatic symptomology will be associated with greater physiological responses to emotionally arousing stimuli.
  \item
    Distinct dimensions of physical sensations and physiological responding will be positively correlated.
  \item
    Gastrointestinal symptoms will be associated with both greater anxiety and greater EGG response to emotionally arousing stimuli.
  \end{itemize}
\end{enumerate}

\bibliography{book.bib,packages.bib}


\end{document}
